%! Author = Martin Vandenbussche
In this thesis, we continued the work of Jean-Pacifique Mbonyincungu on the \textit{NewOz} syntax, which is an attempt to implement the numerous paradigms supported by \textit{Oz}, in a more modern and user-friendly syntax.
We systematically reviewed each class of instructions, providing refinements or modifications to create a new version of \textit{Oz}'s syntax we called \textit{NewOz 2021}.
We then proceeded to create a compiler for this syntax called \texttt{nozc}, with a strong focus on code maintainability and adhesion to open-source principles.
In a third phase we asked for the help of the online community to give us their opinions on the newly designed syntax.
We held a discussion online for a couple of weeks, and compiled this feedback for potential future contributors.
We then gave a critical evaluation of this feedback process, and of the results it delivered.
Finally, we conducted a broader reflection on the objectives behind this new version of \textit{Oz}, and how its purposes could be achieved by porting the principles that govern the language, other existing platforms.\newline

This last idea places \textit{Oz} at a crossroad in its story, and forces to think about how we want to define the language itself.
Do we find the language itself more important than the principles it teaches ?
Do we think of it as a unified and inseparable block, or would we allow a subset of its features to be ported elsewhere ?
Do we find value in the \textit{Mozart} platform itself such that we want to keep using it, or are we ready to let \textit{Oz} and its philosophy find a new home ?
No matter the opinion of each reader on these questions, we feel like this work provided satisfactory advancements in the reflection on the subject.\newline

In the former case, which we could describe as the "conservative standpoint", the improvements made in \textit{NewOz 2021}, and the first versions of \texttt{nozc} provide a solid base for the new syntax to keep growing as a part of the \textit{Mozart} ecosystem.
Future works could provide improvements to the syntax and to \texttt{nozc}, in order to cover more parts of the \textit{Oz} language than those described in~\cite{van2004concepts}.
The integration and compatibility with the existing EMACS environment should also be a priority, given its central part in the \textit{Mozart} ecosystem.\newline
In the latter case, (the "revolutionary view"), we provided arguments for the creation of conservative extensions of existing language(s) providing \textit{Oz}'s approach on specific programming paradigms.
We showed that this process has already been successfully conducted in the past, and that it can provide the same educational value as the existing approach advocated by \textit{Oz} ion its current form.\newline

But even more importantly, the work proved how difficult it is to create a new syntax.
This process takes a lot of time, and requires the input of many people;
no matter which of the stances described above is taken, more feedback periods will be necessary to achieve satisfactory results.
In particular, more attention will need to be paid to the feedback procedure itself, to avoid repeating the mistakes we made and described in section~\ref{sec:ch4-evaluation}.
An interesting option could be to use the new language in a university course, in the context of a mid-sized programming project.
The feedback from students would be close to what future users would experience, and be of great value in the iterative design process.\newline
Additionally, in a much later phase, the various teaching materials used for \textit{Oz} will need to be adapted to take the new syntax (and possibly the new platform) into account.\newline

As we said, the task at hand is long and difficult.
It forces us to constantly reevaluate our stance on things, and to be ready to sometimes reconsider our approach altogether.
This is one of the reasons why this project is taking placing over multiple master theses spanning multiple years;
this work was only a step in the long and ambitious project to create a new, improved, and accepted syntax for the \textit{Oz} multip-paradigm language.
In many ways, \textit{Oz} was a pioneer;
other languages like \textit{Scala} followed its ideas to integrate multiple paradigms in a unified and coherent syntax.
And there is no doubt today that modern and futures languages need to have multi-paradigm capabilities.
For those reasons, it is clear to us that \textit{Oz} still has an important role in this play.\newline
Updating the syntax of \textit{Oz} to be able to bring his advanced capabilities into the hands of more programmers will not be an easy task.
But it is exciting to think that this multi-year project might have an impact on the future of programming languages as a whole.