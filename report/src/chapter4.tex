%! Author = Martin Vandenbussche
In this chapter, we will describe the process we put in place to obtain a good evaluation of the syntax proposed in chapter~\ref{ch:2}.
Starting with the approach we followed to gather feedback from various developers from both in and outside our network,
we will then give a critical evaluation of this process, and explain the reasons that pushed us to adjust it.
Finally, we will conclude the chapter by giving a broader reflection on the approach this thesis took, both when it comes to the design and the evaluation of the syntax, but also on the future we envision for \textit{NewOz}.
Our hope is that those reflections will help future contributors select the most appropriate approach in their work, in order to make \textit{NewOz} as successful as possible.

\section{A first approach : gathering community feedback}\label{sec:ch4-GitHub}
Before describing our evaluation approach, it is important to describe what its objectives were, and what a perfect evaluation would have looked like.\newline
One of the main goals of this thesis, as we briefly mentioned in the introduction chapter, was to gather, for the first time, feedback from people unfamiliar to the project.
Specifically, we wanted to collect opinions on the syntax as it stands at this point in time, after two successive years of work on it.
The importance of this process should not be underestimated as, like in other matters, an outsider's opinion often brings a new perspective on things, pointing a finger on what seemed like an unimportant detail, and asking uncomfortable questions that forces us to reevaluate our stance.\newline

A syntax cannot be designed lightly : if it is to stand the test of time, it should be conceived organically, by gathering feedback and adjusting specific elements, in an iterative process that can (and should) take a long time.
This is the best way to obtain a result that satisfies as many people as possible;
in turn, this means it will be used by a lot of programmers \emph{because it suits their needs}.
After all, we have to remember that programming languages exist to solve real issues people face, be it in a professional, or an educational environment;
this is not a purely theoretical exercise designed by some computer scientists to challenge themselves.\newline

In the light of this, our intention was to put together opinions from as many programmers as possible, and we first took the time to carefully design our evaluation process.
Two main issues had to be tackled : (a) contacting those people and sparkling their interest in the project; (b) finding an effective way to gather their opinions, while allowing a real debate to take place between contributors.\newline

The first point was fairly easy to address, and we sent messages through different channels : mailing lists of EPL alumni, private messages to friends working in STEM, as well as through Peter Van Roy's \textit{Twitter} account, on which the message reached a couple of hundreds of people.\newline
The second one demanded a bit more work.
We decided to use the \textit{issues} feature of \textit{GitHub} to host the discussions, for multiple reasons :
\begin{enumerate}
    \item It is a website that tech-savvy people generally trust and know how to use, at least on a basic level;
    \item It is a highly customizable platform, where issues can be categorized with labels, linked with each other, or cited from elsewhere;
    \item \textit{GitHub} is available is all countries, and has taken specific actions to limit the likelihood of it being blocked in certain parts of the world\footnote{Readers interested in this topic can consult the repository at \url{https://github.com/github/gov-takedowns} for an example of such actions};
    \item Most potential contributors will already have a \textit{GitHub} account;
    if not, creating one is free and easy to do.
\end{enumerate}
It made sense to organize this discussion on the \textit{GitHub} repository already hosting the code for the \texttt{nozc} compiler.
However, we first had to create an extensive documentation around the language, with tutorials and code examples, to help contributors get started with \textit{NewOz}.
This documentation is also available on the same repository, at~\cite{NozcGitHub}.

\subsection{Results of this approach}\label{subsec:ch4-results}
In this section, we will provide a rapid summary of contributions we received from the community;
the interested reader can consult (and participate in) the full discussion online.
All of these opinions and ideas were gathered on \textit{GitHub}, as we said, and will remain available at~\cite{NozcGitHub}.
The contributors keep full intellectual credit for their own contribution;
we simply compile them here in a succinct manner for the purposes of the discussion.\newline

A first suggestion that was brought to the table was the addition of a "\texttt{return}" keyword.
It was mentioned how, being only allowed in functions, it would allow the user to keep in mind the differences with procedures.
It could also make the creation of future syntax highlighting tools easier by clearly identifying the last statement of a function.
However, we identified two problems with this suggestion :
\begin{itemize}
    \item Function bodies are expressions in \textit{Oz}; this change would not fit nicely in that perspective, making the body of all methods essentially a statement;
    \item Distinguishing functions from procedures is arguably easy enough, thanks to the use of separate keywords in their respective definition, but also because of the point above.
\end{itemize}
The point about future tooling like syntax highlighting was indeed important to mention;
even though the language is still far from a state where a proper software ecosystem grows around it, it is still important to keep this sort of things in mind from the start.
In this specific case, the absence of a return keyword means that some sort of analysis phase needs to be performed on the program to determine if a phrase is an expression or a statement, which can in turn determine if it is a suitable last phrase in a function.
It does not seem to us to be a big issue, even though we admittedly don't have much knowledge about this type of tools.
Another option could be to copy the behavior of \textit{Scala}, which \emph{seems} to make the "\texttt{return}" keyword optional;
but this comes with its own problems\footnote{See in that regard the answer from user \texttt{dhg} on \textit{StackOverflow} (\url{https://stackoverflow.com/a/12560532}), but also the blog post from Rob Norris at~\cite{returnscala}}.\newline

Another talking point was the way attributes in classes are expressed.
In the current version of the syntax, they must be declared before the opening of the class' scope (see the code examples in Appendix~\ref{sec:appendix-examples}), with a repetition of the "\texttt{attr}" keyword for each new attribute.
This syntax was rightfully deemed redundant, and it was suggested to take inspiration from \textit{C\#}'s properties syntax to design a more elegant approach\footnote{See \url{https://docs.microsoft.com/en-us/dotnet/csharp/programming-guide/classes-and-structs/properties}}.
More simply, declaring them in a sequence, separating them with commas, could be another option, that would be more akin to the way variables and values are declared.\newline

It was also mentioned that the choice of the keywords "\texttt{var}"/"\texttt{val}" might not be most wise, as they look very similar, which might pose problems for some people.
Different options were proposed, like "\texttt{mut}"/"\texttt{val}", "\texttt{let mut}"/"\texttt{let}", or even "\texttt{cel/cell}"/"\texttt{var}", in a witty nod to \textit{Oz}'s cells feature.
A lot of arguments can be made for each of those propositions and others;
a good point here is that it does not impact the language's philosophy in any way, nor its implementation significantly, as modifications to the lexical grammar have minimal consequences on the compiler's implementation.
Changes to it will, however, break compatibility with existing programs, something that has to be kept in mind in future iterations of the syntax.\newline

Additionally, another good point was made regarding a visually confusing syntax element : the labels and features of records.
The current version of the syntax uses an apostrophe "\texttt{'}" in front of the non-capitalized name, which is necessary to make the distinction with variable names or method calls.
However, it was brought to our attention how this could be not only too subtle for people with poor eyesight, but also how it could be confused with strings in general.
While we agree that the current solution is not satisfactory, we could not find a suitable replacement yet : most characters on the keyboard are either already used, or are not standard enough to appear on all keyboard layouts.
A syntax using chevrons was suggested, but we didn't take it on as it would create a lot of problems with \textit{HTML}-like content;
in particular, online documentation and emails displaying code snippets would see their formatting disrupted if these particular characters were used without careful and quite tedious escaping.\newline

Finally, we will quickly mention the problem that was brought to our attention regarding Unicode support in \textit{Mozart}.
Even though \texttt{nozc} could natively support all Unicode characters, the fact that the Mozart2 compiler doesn't, poses a problem : a variable named "\texttt{bµ}" can't be compiled by the underlying \texttt{ozc} compiler.
This problem could be circumvented by "translating" problematic characters, giving something like "\texttt{B\_C2B5\_}" in output of \texttt{nozc}.
We find this solution acceptable since it probably won't enter into play very often;
nonetheless, it is an important addition to include in a future version of \texttt{nozc}.

\section{Evaluation of those results}\label{sec:ch4-evaluation}
Our general takeaway on the feedback we received is the following : \emph{we didn't get the high-level, philosophical reflections we expected, but the fault probably lies in our ill-suited approach for a debate on those subjects.}\newline

In terms of content, we hoped for more content-focused reactions on the general philosophy of the language.
Instead, we mainly got propositions for the usage of a particular keyword or small-scope syntax modifications.
We identify two possible reasons for this discrepancy between the expected and the actual feedback.\newline

First of all, outside users will use the language for a short amount of time before giving feedback.
Granted, we can't reasonably blame them for not willing to invest hours upon hours on contributing to an open source project online, to which they dedicate their time freely.
But this means that the feedback they are able to give is mainly focused on what is apparent at first glance, that is, the "vocabulary" of the syntax.
In-depth reflections can only come after extensive use of the syntax, from people having written different programs using various paradigms.
In that regard, calling upon the online community to help us in a deep reflection on the philosophy of a syntax was probably a process that was doomed to fail.\newline

Secondly, the "philosophy" behind \textit{NewOz} was maybe not explicit enough in the first place;
how then could users react upon it ?
The debate would probably have benefited from a deeper high-level presentation of the language in the documentation, similar to the extensive syntax tutorial that was written.
Such a document could have presented our vision more explicitly, which would have allowed contributors to share some of their opinions on it.\newline

With all this being said, the remarks we did gather still raised interesting questions and will definitely be useful in the design process of \textit{NewOz}.
Relevant syntax elements from different languages were proposed, and it is clear that such proposals are essential to design a good syntax, simply because the experience of each programmer is different, and so is their knowledge and approach of what a powerful, convenient, or even fun programming syntax is.

\section{A second approach : a broader reflection on the project itself}\label{sec:ch4-reflection}
These thoughts encouraged us to take a step back and reevaluate the \textit{NewOz} project as a whole.
Up until now, we always envisioned this new version of the language as a stand-alone system, that would either run on the existing \textit{Mozart} virtual machine or on its own, depending on the choices that will be made in the future regarding the compiler.
However, a new idea was brought up during a discussion with Martin Henz, Associate Professor at the National University of Singapore\footnote{\url{https://www.comp.nus.edu.sg/cs/bio/henz/}} and reader of this thesis.
What if, instead of "refreshing" \textit{Oz} and its concepts in the form of a new syntax, we took the interesting concepts and paradigms behind \textit{Oz}, and incorporated them in an existing language with a robust syntax and more importantly, an established community ?
This approach would have many advantages :
\begin{itemize}
    \item By creating an extension to an existing language, we could easily reach its existing developers community;
    \item We would build upon an existing syntax, which has already gone through the long process of design and acceptance;
    \item We could also probably reuse the existing software ecosystem around the language, at least to some extent;
    \item It would give more credibility and visibility to the language than the "stand-alone" approach, making it easier to attract and engage newcomers.
\end{itemize}
Amazingly, this addresses most of the problems \textit{Oz} faces today, that we described in chapter~\ref{ch:1}.\newline

The attentive reader might notice how this approach is looks like the one \textit{Ozma} selected years ago : by creating a conservative extension to \textit{Scala}, this project was able to incorporate powerful concepts from \textit{Oz} into an existing syntax~\cite{Ozma}.
However, there is one key difference that need to be explained : \textit{Ozma} did not target an existing platform.
In particular, it does not target the \textit{JVM} : the \textit{Ozma} compiler uses the \textit{Mozart} engine as its backend, even though its syntax is an extension of Scala's.
The approach we propose here is somewhat different, in the sense that we would target the same platform as the base language of our conservative extension.\newline

Another work which followed a similar approach is \textit{Flow Java}~\cite{drejhammar2003flow}, which conservatively extended \textit{Java} and the \textit{Gnu Java Compiler} to support "single assignment variables and futures as variants of logic variables".
This work targeted the \texttt{libjava} runtime environment.\newline

Driven by his long teaching experience at the NUS, Martin Henz proposed the idea to create a conservative extension of \textit{JavaScript}, which could address a number of points that are considered problematic in the current version of the language and in its implementations :
\begin{itemize}
    \item The mystically-called "Temporal Dead Zone" (or TDZ), which requires a bit of an explanation.
    In \textit{JavaScript}, variables can be declared in 3 ways : (a) using \texttt{var}, which makes the variable global (if it is declared outside a function), or function scoped otherwise; (b) using \texttt{let}, which makes the variable block-scoped (a block is defined as anything between curly braces \texttt{\{\}}); (c) using \texttt{const}, which essentially follows the same rules as \texttt{let} with the exception that the variable is -you guessed it- a constant, making it only assignable once.
    But there is another important, albeit more subtle difference that is important to understand when choosing the correct keyword, which has to do with a concept called \textit{variable hoisting}.\newline
    Let us take an example to explain this concept (see Figure~\ref{fig:js-hoisting}\footnote{The code examples in this section were taken from~\cite{jsTDZ}}).
    \begin{figure}[H]
    \begin{lstlisting}[language=JavaScript]
    {
        console.log(hoistedVariable); // undefined
        var hoistedVariable = 1;
    }
    // Which is functionally identical to :
    {
        var hoistedVariable; // Initialized to undefined
        console.log(hoistedVariable); // undefined
        hoistedVariable = 1;
    }
    \end{lstlisting}
    \caption{Hoisting of a \texttt{var}}
    \label{fig:js-hoisting}
    \end{figure}
    As you can see, when \textit{hoisting} a \texttt{var}, the parser gives it an \texttt{undefined} value.
    Compare now with the following example (see Figure~\ref{fig:js-hoisting-2}):
    \begin{figure}[H]
    \begin{lstlisting}[language=javaScript]
    {
        console.log(hoistedVariable); // undefined
        let hoistedVariable = 1;
    }
    // Which is functionally identical to :
    {
        let hoistedVariable; // Not initialized
        // hoistedVariable is in the TDZ
        console.log(hoistedVariable); // Throws a ReferenceError
        hoistedVariable = 1;
    }
    \end{lstlisting}
    \caption{Hoisting of a \texttt{let}}
    \label{fig:js-hoisting-2}
    \end{figure}
    \textit{Hoisting} refers to the apparent rearrangement of arguments done by the \textit{JS} parser.
    The idea behind this, is that variables declared in a scope are made accessible at \emph{any point in said scope}.
    Because of this design choice to allow variables access \emph{before} their declaration, the \textit{hoisting} process is required for the parsing to succeed.
    In most languages, this is not allowed : in \textit{Oz} for example, the declarations have to appear before their use in any given scope.
    The implementation of \texttt{var}s \textit{hoisting} in \textit{JavaScript} is often considered problematic\footnote{See~\cite{whyTDZ}}, because it hides errors.
    If a programmer declared a variable after accessing it, it was probably by accident : for this reason, they should receive a warning from the parser.
    The implementation of \textit{hoisting} for \texttt{var}s, however, does not throw any warning, since the variable is silently initialized to \texttt{undefined}.
    \textit{ES6} corrects this problem with the TDZ, which describes the period between a variable (\texttt{let} or \texttt{const}) declaration and its initialization.
    Any access attempt during this period will result in a \texttt{ReferenceError}.\newline

    This is a nice approach, but it still can be confusing for the programmer, as it isn't very intuitive : if I am not allowed to access the variable, why could I declare it after its access in the first place ?
    \textit{Oz} provides a much more elegant approach, with the concept of \textit{unbound values} (see Figure~\ref{fig:oz-unbound}).
    Any function that needs to access such a value will wait until it is available, which allows creating data-driven operations (\textit{dataflow execution}).
    This is a fundamentally different approach to this problem, but one that is probably worth exploring in an extension of \textit{JavaScript}, notably by using its asynchronous capabilities\cite{ozjs}.
    \begin{figure}
        \begin{lstlisting}[language=oz]
        local
            A % Equivalent to A = _
        in
            {Browse {PerformComputation 5 1 A}}
            % The function holds until A gets a value
            % A is still unbound
            A = 1 % The function can get evaluated
        end
        \end{lstlisting}
        \caption{Unbound values in \textit{Oz}}
        \label{fig:oz-unbound}
    \end{figure}
    \item The absence of proper multi-threading capabilities.
    The \textit{JavaScript} runtime uses an event-driven system which sequentially executes actions based on events occurring on the web page.
    This gives the impression of a concurrent system, but it is far less powerful than the declarative concurrency capabilities \textit{Oz} offers.
    It would thus be interesting to explore an extension of \textit{JavaScript} allowing to simulate the behavior of threads in a syntax as neat and concise as \textit{Oz}'s.
    This could be achieved using \textit{web workers}, or using asynchronous functions, as demonstrated in~\cite{ozjs}.
    \item A third problem with \textit{JavaScript} lies in the fact that most \textit{JavaScript} engines don't implement tail call optimization in function.
    Because functions have to return a value (\texttt{undefined} in the case of procedures), even a well-written, tail-recursive function written by a smart programmer using an accumulator will not see the memory optimization they could expect, because of the presence of this \texttt{undefined} value on the stack.
    The ECMAScript 2015 specification includes optimizations for this specific case\footnote{See \url{https://262.ecma-international.org/6.0/\#sec-tail-position-calls}}.
    Sadly, very few platforms have implemented them at the time of writing\footnote{See \url{https://kangax.github.io/compat-table/es6/}. Some vendors are actually saying they don't even plan on supporting the feature (\url{https://www.chromestatus.com/feature/5516876633341952\#details})}.
    The lack of support by vendors is once again a great opportunity to demonstrate the power of \textit{Oz}'s logic variables, which would bypass the need to maintain a memory-hungry call stack.
\end{itemize}
The list goes on, but we feel like those examples are enough to show how interesting it could be to integrate concepts of \textit{Oz} into other languages.
Future could design conservative extensions of existing languages adding the abilities of one or two paradigms at a time.
This approach would fit very nicely in many programming courses, by introducing students to a few concepts at the start, and then adding paradigms on top of acquired concepts.
The educational value of this approach is invaluable, and it is used in university courses like the ones given by Martin Henz and Peter van Roy.
It is also the approach followed in~\cite{van2004concepts}.\newline

Clearly, this new idea is drastically different from what was envisioned at the start of this thesis.
But we feel like it combines a lot of the advantages over a complete rewrite of \textit{Oz}, on top of the benefits and stability brought by the integration into an existing platform.
Next steps in this direction should now make the difficult decision of choosing the target language/platform that would receive those conservative extensions.
We already made a good case for \textit{JavaScript}, but works like~\cite{Ozma} and~\cite{drejhammar2003flow} show that the \textit{JVM} is also a perfectly valid target.
Arguments for the selection of a platform should evaluate elements like the size and engagement of the community, the philosophy and future objectives of the institution promoting it, as well as open-source and legal considerations, which are important since the platform would be modified and used in an educational environment.
