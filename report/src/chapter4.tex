%! Author = Martin Vandenbussche

In this chapter, we will describe the process we put in place to obtain a good evaluation of the syntax proposed in chapter~\ref{ch:2}.
Starting with the approach we followed to gather feedback from various developers from both in and outside our network,
we will then give a first critical evaluation of this process for gathering feedback, and explain the reasons that pushed us to adjust it in a second phase.
Finally, we will conclude the chapter by giving a broader reflection on the approach this thesis took, both when it comes to the design and the evaluation of the syntax, but also on the future we envision for \textit{NewOz}.
Our hope is that those reflections will help future contributors select the most appropriate approach in their work, in order to make \textit{NewOz} as successful as possible.

\section{A first approach : gathering community feedback}\label{sec:ch4-GitHub}
Before describing our evaluation approach, it is important to describe what its objectives were, and what a perfect evaluation would have looked like.\newline
One of the main goals of this thesis, as we briefly mentioned in the introduction chapter, was to gather, for the first time, feedback from people unfamiliar to the project.
Specifically, we wanted to collect opinions on the syntax as it stands at this point in time, after two successive years of work on it.
The importance of this process can't be overstated as, like in other matters, an outsider's opinion often brings a new perspective on things, pointing a finger on what seemed like an unimportant detail, and asking uncomfortable questions that forces us to reevaluate our stance.\newline

A syntax can't be designed lightly : if it is to stand the test of time, it should be conceived organically, by gathering feedback and adjusting specific elements, in an iterative process that can (and should) take a long time.
This is the best way to obtain a result that satisfies as many people as possible;
in turn, this means it will be used by a lot of programmers \emph{because it suits their needs}.
After all, we have to remember that programming languages exist to solve real issues people face, be it in a professional or an educational environment;
this is not a purely theoretical exercise designed by some computer scientists to challenge themselves.\newline

In the light of this, our intention was to put together opinions from as many programmers as possible, and we first took the time to carefully design our evaluation process.
Two main issues now had to be tackled : (a) contacting those people and sparkling their interest in the project; (b) finding an effective way to gather their opinions, while allowing a real debate to take place between contributors.\newline

The first point was fairly easy to address, and we send messages through different channels : mailing lists of EPL alumni, private messages to friends working in STEM, as well as Professor Van Roy's \textit{Twitter} account, on which the message reached a couple of hundreds of people.\newline
The second one demanded a bit more work.
We decided to use the \textit{issues} feature of \textit{GitHub} to host the discussions, for multiple reasons :
\begin{enumerate}
    \item It is a website that tech-savvy people generally trust and know how to use, at least on a basic level;
    \item It is a highly customizable platform, where issues can be categorized with labels, linked with each other, or cited from elsewhere;
    \item \textit{Github} is available is all countries, and has taken specific actions to limit the likelihood of it being blocked in certain parts of the world\footnote{Readers interested in this topic can consult the repository at \url{https://github.com/github/gov-takedowns} for an example of such actions};
    \item Most potential contributors will already have a \textit{GitHub} account;
    if not, creating one is free and easy to do.
\end{enumerate}
It made sense to host this discussion on the \textit{GitHub} repository already hosting the code for the \texttt{nozc} compiler.
However, we firstly had to create an extensive documentation around the language, with tutorials and code examples, to help contributors get started with \textit{NewOz}.
This documentation is also available on the same repository, at~\cite{NozcGitHub}.

\subsection{Results of this approach}
In this section, we will provide a summary of contributions we received from the community.
All of these opinions and ideas were gathered on \textit{GitHub}, as we said, and will remain available online at~\cite{NozcGitHub}.
The contributors keep full intellectual credit for their contribution;
we simply compile them here in a succinct manner for the purposes of the discussion.\newline

(a) return keyword (b) attribute pos in classes (c) val and var (d) records definition (e) unicode adoption\newline
For each, briefly and if applicable :
\begin{itemize}
    \item Description of the change
    \item Motivations (previous problem, how it fixes it, philosophy of \textit{Oz}) : personal opinion
    \item Implications on other existing features
    \item Implications in the compiler
    \item What did others think of it ? (probably =/= my opinion) Should we integrate their feedback ? Why/why not ?
\end{itemize}

\section{First evaluation and adjustments}\label{sec:ch4-adjustments}
What was user feedback in general ?
First impressions of newcomers (relevant for forging our expectations on what future students will say, for example).\newline
Can we say this feedback met our goal described in \textit{NewOz}'s philosophy ?
Not in terms of numbers.
In terms of content, we hoped for "deeper"/"higher-level" reflections.
Instead, we mainly got propositions for the usage of a particular keyword or small-scope syntax modifications.\newline
We identify two possible reasons for this discrepancy between the expected and the actual feedback.\newline

First of all, outside users will use the language for a short amount of time before giving feedback.
Granted, we can't reasonably blame them for not willing to invest hours upon hours on contributing to an open source project online, to which they dedicate some time freely.
But this means that the feedback they are able to give is mainly focused on what is apparent at first glance, that is, the "vocabulary" of the syntax.
In-depth reflections can only come after extensive use of the syntax, from people having written different programs using various programming paradigms.
In that regard, calling upon the online community to help us in a deep reflection on the philosophy of a syntax was probably a process that was doomed to fail.\newline

Nevertheless, the remarks we did gather raised interesting questions and will definitely be useful in the design process of \textit{NewOz}.
Relevant syntax elements from different languages were proposed, and it is clear that such proposals are essential to design a good syntax, simply because the experience of each programmer is different, and so is their knowledge and approach of what a powerful, convenient, or even fun programming syntax is.

\section{A second approach : a broader reflection on the project itself}\label{sec:ch4-reflection}
\begin{itemize}
\item Nous sommes une étape dans travail de longue haleine (long-term project/collaboration) - spread over multiple master's projects
\item Toward ultimate goal of a new, improved, and accepted syntax for the Oz multi-paradigm language (Multiparadigm = now it is accepted that languages must be multiparadigm-Java has lambdas, Scala is functional-objet, Cloud analytics combine functional, concurrent, and database structure)
\item Etre honnête - c'est compliqué de faire une syntaxe, les limites de temps rencontrées, pourquoi c'est un process multi-year -> ce qu'on a fait pour pallier à ces limitations "physiques"
\item Final, definitive way of formulating a multiparadigm language.  Oz was a pioneer, followed by Scala, Ozma, etc., but what will multiparadigm languages look like in the future?  In the future when all languages are multiparadigm. We are making steps toward this - take Oz original ideas but with new syntax inspired by existing languages. Lyric goal : here we made one small step in this (Appollo reference)
\end{itemize}
