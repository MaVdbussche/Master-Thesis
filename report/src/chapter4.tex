%! Author = Martin Vandenbussche

In this chapter, we will describe the process we put in place to obtain a good evaluation of the syntax proposed in chapter~\ref{ch:2}.
Starting with the [things] we put in place to gather feedback from various developers  from both in and outside our network,
we will then give a first critical evaluation of this approach for gathering feedback, and explain the reasons that pushed us to adjust it in a second phase.
Finally, we will conclude the chapter by giving a broader reflection on the approach this thesis took, both when it comes to the design and the evaluation of the syntax, but also on the future we envision for \textit{NewOz}.
Our hope is that those reflections will help future contributors select the most appropriate approach in their work, in order to make \textit{NewOz} as successful as possible.
 \begin{itemize}
     \item qu'aurait été une évaluation parfaite ?
     \item ce qu'on a pu faire
     \item le feedback qu'on a eu
     \item Notre évaluation (Martin, Peter, Martin)
 \end{itemize}
\section{A first approach : gathering community feedback}\label{sec:ch4-GitHub}
This chapter = general community feedback - résumé des suggestions.\newline
Describe how we reached potential contributors, GitHub issues mgmt, repository, etc.
What was the objective with this feedback ?

"We will now describe the feedback we received on GitHub"
\subsection{}
Describe changes that were proposed to the \textit{NewOz} syntax.
For each :
\begin{itemize}
    \item Description of the change
    \item Motivations (previous problem, how it fixes it, philosophy of \textit{Oz}) : personal opinion
    \item Implications on other existing features
    \item Implications in the compiler
    \item What did others think of it ? (probably =/= my opinion) Should we integrate their feedback ? Why/why not ?
\end{itemize}

\section{Première évaluation et ajustement de l'approche}\label{sec:ch4-adjustments}
What was user feedback in general ?
First impressions of newcomers (relevant for forging our expectations on what future students will say, for example).\newline
Can we say this feedback met our goal described in \textit{NewOz}'s philosophy ?
Not in terms of numbers.
In terms of content, we hoped for "deeper"/"higher-level" reflections.
Instead, we mainly got propositions for the usage of a particular keyword or small-scope syntax modifications.\newline
We identify two possible reasons for this discrepancy between the expected and the actual feedback.\newline

First of all, outside users will use the language for a short amount of time before giving feedback.
Granted, we can't reasonably blame them for not willing to invest hours upon hours on contributing to an open source project online, to which they dedicate some time freely.
But this means that the feedback they are able to give is mainly focused on what is apparent at first glance, that is, the "vocabulary" of the syntax.
Content-focused ["de fond"] reflections can only come after extensive use of the syntax, after writing different programs using various programming paradigms.
In that regard, calling upon the online community to help us in a deep reflection on the [approach] for a syntax was probably an approach that was doomed to fail.\newline

Nevertheless, the remarks we did gather raised interesting questions and will definitely be useful in the design process of \textit{NewOz}.
Relevant syntax elements from different languages were proposed, and it is clear that such proposals are essential to design a good syntax, simply because the experience of each programmer is different, and so is their knowledge and approach of what a powerful, convenient, or event fun programming syntax is.

\section{A second approach : a broader reflection on the project itself}\label{sec:ch4-reflection}