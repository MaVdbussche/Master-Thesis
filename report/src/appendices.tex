%! Author = Martin Vandenbussche
\appendix
\chapter*{Appendices}
\addcontentsline{toc}{chapter}{Appendices}
\renewcommand{\thesection}{\Alph{section}}

\section{\textit{NewOz} EBNF grammar (2021 version)}\label{sec:appendix-grammar}
This EBNF grammar is a reworked version of the one provided in the appendices of Jean-Pacifique Mbonyincungu's thesis,
removing left-recursion problems and including changes made in the syntax since then.
It is thus suitable for recursive descent, and is the grammar used by \texttt{nozc}.
%! Author = Martin Vandenbussche
\begin{lstlisting}[label={lst:newOzEBNF},language=ebnf]
Note that the concatenation symbol in EBNF (comma) is
omitted for readability reasons
Notation      Meaning
===========================================================================
;\epsilon;             singleton containing the empty word
;$(w)$;           grouping of regular expressions
;$[w]$;           union of ;\epsilon; with the set of words ;$w$; (optional group)
;$\{w\}$;          zero or more times ;$w$;
;$\{w\}+$;         one or more times ;$w$;
;$w_1~w_2$;        concatenation of ;$w_1$; with ;$w_2$;
;$w_1 | w_2$;         logical union of ;$w_1$; and ;$w_2$; (OR)
;$w_1-w_2$;       difference of ;$w_1$; and ;$w_2$;

interStatement ::= {nestConStatement}+
            | DECLARE inStatement EOF

statement ::= nestConStatement
            | SKIP

expression ::= nestConExpression
            | nestDecAnonym
            | DOLLAR
            | term
            | THIS
            | inExpression

parExpression ::= LPAREN expression RPAREN

//Declarations still need to come first (keeps Oz' idea)
inStatement ::= LCURLY {declarationPart} {statement} RCURLY

inExpression ::= LCURLY {declarationPart} {statement} [expression] RCURLY

nestConStatement ::= assignmentExpression
            | variable LPAREN [expression {COMMA expression}] RPAREN
            | inStatement
            | IF parExpression inStatement
              {ELSE IF LPAREN expression RPAREN inStatement}
              [ELSE inStatement]
            | MATCH expression LCURLY
                {CASE caseStatementClause}+
                [ELSE inStatement]
              RCURLY
            | FOR LPAREN {loopDec}+ RPAREN inStatement
            | TRY inStatement
              [CATCH LCURLY
                {CASE caseStatementClause}+
              RCURLY]
              [FINALLY inStatement]
            | RAISE inExpression
            | THREAD inStatement
            | LOCK [LPAREN expression RPAREN] inStatement

nestConExpression ::= IF LPAREN expression RPAREN inExpression
                {ELSE IF LPAREN expression RPAREN inExpression}
                [ELSE inExpression]
            | MATCH expression LCURLY
               {CASE caseExpressionClause}+
               [ELSE inExpression]
              RCURLY
            | TRY inExpression
              [CATCH LCURLY
                {CASE caseExpressionClause}+
              RCURLY]
              [FINALLY inStatement]
            | RAISE inExpression
            | THREAD inExpression

nestDecVariable ::= DEFPROC variable LPAREN [pattern {COMMA pattern}] RPAREN inStatement
            | DEF [LAZY] variable LPAREN [pattern {COMMA pattern}] RPAREN inExpression
            | FUNCTOR [variable] {
                (IMPORT importClause {COMMA importClause}+)
                | (EXPORT exportClause {COMMA exportClause}+)
              }
              inStatement
            | CLASS variableStrict {classDescriptor} LCURLY
              {classElementDef} RCURLY

nestDecAnonym ::= DEFPROC DOLLAR LPAREN [pattern {COMMA pattern}] RPAREN inStatement
            | DEF [LAZY] DOLLAR LPAREN [pattern {COMMA pattern}] RPAREN inExpression
            | FUNCTOR [DOLLAR] {
                (IMPORT importClause {COMMA importClause}+)
                | (EXPORT exportClause {COMMA exportClause}+)
              }
              inStatement
            | CLASS DOLLAR {classDescriptor} LCURLY
              {classElementDef} RCURLY

importClause ::=  variable
                    [LPAREN (atom|int)[COLON variable] {COMMA (atom|int)[COLON variable]} RPAREN]
                    [FROM atom]

exportClause ::= [(atom|int) COLON] variable

classElementDef ::= DEFPROC methHead [ASSIGN variable] (inExpression|inStatement)
            | classDescriptor

caseStatementClause ::= pattern {(LAND|LOR) conditionalExpression} IMPL inStatement

caseExpressionClause ::= pattern {(LAND|LOR) conditionalExpression} IMPL inExpression

assignmentExpression ::= conditionalExpression

assignmentStatement ::= variable ASSIGN expression

conditionalExpression ::= conditionalOrExpression

conditionalOrExpression ::= conditionalAndExpression {LOR conditionalAndExpression}

conditionalAndExpression ::= equalityExpression {LAND equalityExpression}

equalityExpression ::= relationalExpression {EQUAL relationalExpression}

relationalExpression ::= additiveExpression [(GT|GE|LT|LE) additiveExpression]

additiveExpression ::= multiplicativeExpression {(PLUS|MINUS) multiplicativeExpression}

multiplicativeExpression ::= unaryExpression {(STAR|SLASH|MODULO) unaryExpression}

unaryExpression ::= (MINUS|PLUS) unaryExpression
            | simpleUnaryExpression

simpleUnaryExpression ::= LNOT unaryExpression
            | postfixExpression

postfixExpression ::= primary

primary ::= variable | int | float | character | string | UNIT | TRUE | FALSE | UNDERSCORE | NIL
            | variable LPAREN [expression {COMMA expression}] RPAREN
            | variable DOT variable [LPAREN [expression {COMA expression}] RPAREN]
            | THIS DOT variable LPAREN [expression {COMMA expresssion}] RPAREN
            | SUPER [LPAREN variableStrict RPAREN]
                DOT variable LPAREN [expression {COMMA expression}] RPAREN
            | parExpression

term ::= assignmentExpression
            | atomLisp LPAREN [[feature COLON]pattern {COMMA [feature COLON]pattern} [COMMA ELLIPSIS]] RPAREN
            | LPAREN expression {HASHTAG expression}+ RPAREN
            | LPAREN expression {COLCOL expression}+ RPAREN
            | LBRACK [expression {COMMA expression}] RBRACK

pattern ::= variable | int | float | character | string | UNIT | TRUE | FALSE | UNDERSCORE | NIL
            | atomLisp LPAREN [[feature COLON]pattern {COMMA [feature COLON]pattern} [COMMA ELLIPSIS]] RPAREN
            | LPAREN pattern {HASHTAG pattern}+ RPAREN
            | LPAREN pattern {COLCOL pattern}+ RPAREN
            | LBRACK [pattern {COMMA pattern}] RBRACK
            | LPAREN pattern RPAREN

declarationPart ::= (VAL|VAR) variable [ASSIGN expression]
                      {COMMA variable [ASSIGN expression]}
            | nestDecVariable

loopDec ::= variable IN expression DOTDOT expression [SEMI expression]
            | variable IN expression SEMI expression [SEMI expression]
            | variable IN expression

feature ::= atomLisp

classDescription ::= EXTENDS variableStrict {COMMA variableStrict}+
            | ATTR variable [ASSIGN expression]

methHead ::= (variableStrict | variable)
              LPAREN [methArg {COMMA methArg} [COMMA ELLIPSIS]] RPAREN

methArg ::= [feature COLON] (variable | UNDERSCORE) [LE expression]

\end{lstlisting}


\section{Lexical grammar (2021 version)}\label{sec:appendix-lexical-grammar}
This is the lexical grammar used by \texttt{nozc}.
%! Author = Martin Vandenbussche
\begin{lstlisting}[label={lst:newOzLexical},language=ebnf]
Notation      Meaning
===========================================================================
;\epsilon;             singleton containing the empty word
;$(w)$;           grouping of regular expressions
;$[w]$;            union of ;\epsilon; with the set of words ;$w$; (optional group)
;$\{w\}$;           zero or more times ;$w$;
;$\{w\}+$;         one or more times ;$w$;
;$w_1~w_2$;         concatenation of ;$w_1$; with ;$w_2$;
;$w_1 | w_2$;         logical union of ;$w_1$; and ;$w_2$; (OR)
;$w_1-w_2$;       difference of ;$w_1$; and ;$w_2$;

// White spaces - ignored
WHITESPACE ::= (" "|"\b"|"\t"|"\n"|"\r"|"\f")

// Comments - ignored
("//" {~("\n"|"\r")} ("\n"|"\r"|"\r\n")
// Multi-line comments - ignored
"/*" {CHAR - "*/"} "*/"

// Reserved keywords
AT      ::= "at"
ATTR    ::= "attr"
BREAK   ::= "break"
CASE    ::= "case"
CATCH   ::= "catch"
CLASS   ::= "class"
CONTINUE::= "continue"
DECLARE ::= "declare"
DEF     ::= "def"
DEFPROC ::= "defproc"
DO      ::= "do"
ELSE    ::= "else"
EXPORT  ::= "export"
EXTENDS ::= "extends"
FALSE   ::= "false"
FINALLY ::= "finally"
FOR     ::= "for"
FROM    ::= "from"
FUNCTOR ::= "functor"
IF      ::= "if"
IMPORT  ::= "import"
IN      ::= "in"
LAZY    ::= "lazy"
LOCK    ::= "lock"
MATCH   ::= "match"
NIL     ::= "nil"
OR      ::= "or"
PROP    ::= "prop"
RAISE   ::= "raise"
RETURN  ::= "return"
SKIP    ::= "skip"
SUPER   ::= "super"
THIS    ::= "this"
THREAD  ::= "thread"
TRUE    ::= "true"
TRY     ::= "try"
UNIT    ::= "unit"
VAL     ::= "val"
VAR     ::= "var"

// Operators
ASSIGN      ::= "="
PLUSASS     ::= "+="
MINUSASS    ::= "-="
EQUAL       ::= "=="
NE          ::= "\\="
LT          ::= "<"
GT          ::= ">"
LE          ::= "<="
GE          ::= ">="
IMPL        ::= "=>"
LAND        ::= "&&"
LOR         ::= "||"
LNOT        ::= "!"
MINUS       ::= "-"
PLUS        ::= "+"
STAR        ::= "*"
SLASH       ::= "/"
MODULO      ::= "%"
HASHTAG     ::= "#"
UNDERSCORE  ::= "_"
DOLLAR      ::= "$"
APOSTROPHE  ::= "'"
QUOTE       ::= "\""
DEGREE      ::= "°"
COLCOL      ::= "::"
COMMA       ::= ","
LBRACK      ::= "["
LCURLY      ::= "{"
LPAREN      ::= "("
RBRACK      ::= "]"
RCURLY      ::= "}"
RPAREN      ::= ")"
SEMI        ::= ";;"
COLON       ::= ":"
DOT         ::= "."
DOTDOT      ::= ".."
ELLIPSIS    ::= "..."

// Literals
VARIABLESTRICT ::= UPPERCASE{ALPHANUM}
                    | "`"(ESC | PSEUDO_CHAR | ~("`"|"\\"|"\n"|"\r") )"`")
VARIABLE       ::= LOWERCASE{ALPHANUM}
ATOM           ::= (ATOMLISP | "´" (ESC | PSEUDO_CHAR | ~("\\"|"\n"|"\r") ) "´")
ATOMLISP       ::= "'" {LETTER}
STRING         ::= "\"" { ESC | PSEUDO_CHAR | ~("\""|"\\"|"\n"|"\r") } "\""
CHARACTER      ::= (DEGREE(CHARCHAR | PSEUDO_CHAR)
                    | "'" (ESC | ~("'"|"\\"|"\n"|"\r") ) "'" )
INT            ::= (DECINT | HEXINT | OCTINT | BININT)
FLOAT          ::= {DIGIT}+ DOT {DIGIT} [ ("e"|"E")["~"]{DIGIT}+ ]
UPPERCASE      ::= "A"|...|"Z"
LOWERCASE      ::= "a"|...|"z"
LETTER         ::= "A"|...|"Z"|"a"|...|"z"
DIGIT          ::= "0"|...|"9"
NON_ZERO_DIGIT ::= "1"|...|"9"
CHARINT        ::= ("0"|...|"9") | ("1"|...|"9")("0"|...|"9")
                    | "1"("0"|...|"9")("0"|...|"9")
                    | "2"("0"|...|"4")("0"|...|"9")|"25"("0"|...|"5") // (0-255)
ALPHANUM       ::= (UPPERCASE | LOWERCASE | DIGIT | UNDERSCORE)
DECINT         ::= ("0" | (NON_ZERO_DIGIT{DIGIT}))
HEXINT         ::= "0" ("x"|"X") {HEXDIGIT}+
OCTINT         ::= "0" {OCTDIGIT}+
BININT         ::= "0" ("b"|"B") {BINDIGIT}+
OCTDIGIT       ::= "0"|...|"7"
HEXDIGIT       ::= (DIGIT | ("A"|...|"F") | ("a"|...|"f"))
BINDIGIT       ::= ("0"|"1")
ESC            ::= "\\" ESCAPE_CHAR
ESCAPE_CHAR    ::= ("a"|"b"|"f"|"n"|"r"|"t"|"\\"|"'"|"\""|DEGREE)
CHARCHAR       ::= ~("\\")
// In the classes of words <variable>, <atom>, <string>, and <character>, we use pseudo-characters, which represent single characters in different notations.
PSEUDO_CHAR    ::= ( "\\"(OCTDIGIT)(OCTDIGIT)(OCTDIGIT) ) | ( "\\"("x"|"X")(HEXDIGIT)(HEXDIGIT) )
// End of file
EOF            ::= "<end of file>"

\end{lstlisting}

\section{Kernel language}\label{sec:appendix-kernel}
The following table~\ref{tab:OzKernel} gives the kernel language of the \textit{general computational model} of \textit{Oz}, as defined in~\cite{van2004concepts}.
The right-most column gives the \textit{NewOz} equivalent, which allows the user to quickly compare both approaches.
%! Author = Martin Vandenbussche
\begin{table}[H]
\begin{adjustbox}{width=1.2\textwidth,center}
\begin{tabular}{|lr|ll|}
\hline
<s> ::= & & & <s> ::=\\
~~~~\textbf{skip} & \multicolumn{2}{c}{Empty statement} & ~~~~\textbf{skip}\\
~~|  <s>\textsubscript{1} <s>\textsubscript{2} & \multicolumn{2}{c}{Statement sequence} & ~~|  <s>\textsubscript{1} <s>\textsubscript{2}\\
~~|  \textbf{local} <x> \textbf{in} <s> \textbf{end} & Variable creation & Value declaration & ~~|  \{ \textbf{val} <x> <s> \}\\
~~|  <x>\textsubscript{1}=<x>\textsubscript{2} & Variable binding & Variable/value binding & ~~|  <x>\textsubscript{1}=<x>\textsubscript{2}\\
~~|  <x>=<v> & Value creation & Variable/Value assignment & ~~|  <x>=<v>\\
\hline
~~|  \{<x> <y>\textsubscript{1} \ldots <y>\textsubscript{n}\} & \multicolumn{2}{c}{Procedure application} & ~~|  x(<y>\textsubscript{1}, \ldots, <y>\textsubscript{n})\\
\begin{tabular}[c]{@{}l@{}}~~|  \textbf{if} <x> \textbf{then} <s>\textsubscript{1}\\~~~~\textbf{else} <s>\textsubscript{2} \textbf{end}\end{tabular} & \multicolumn{2}{c}{Conditional} & \begin{tabular}[c]{@{}l@{}}~~|  \textbf{if} (<x>) \{ <s>\textsubscript{1} \}\\~~~~\textbf{else} \{ <s>\textsubscript{2} \}\end{tabular}\\
\begin{tabular}[c]{@{}l@{}}~~|  \textbf{case} <x> \textbf{of} <pattern>\\~~~~\textbf{then} <s>\textsubscript{1}\\~~~~\textbf{else} <s>\textsubscript{2} \textbf{end}\end{tabular} & \multicolumn{2}{c}{Pattern matching} & \begin{tabular}[c]{@{}l@{}}~~|  \textbf{match} <x> \{\\~~~~\textbf{case} <pattern> => \{<s>\textsubscript{1}\}\\~~~~\textbf{else} \{ <s>\textsubscript{2} \} \}\end{tabular}\\
~~|  \textbf{thread} <s> \textbf{end} & \multicolumn{2}{c}{Thread creation} & ~~|  \textbf{thread} \{ <s> \}\\
~~|  \{ByNeed <x> <y>\} & \multicolumn{2}{c}{Trigger creation} & ~~|  byNeed(<x>, <y>)\\
\hline
~~|  \{NewName <x>\} & \multicolumn{2}{c}{Name creation} & ~~| newName(<x>)\\
~~|  <y>=!!<x> & \multicolumn{2}{c}{Read-only view} & \emph{Not implemented yet}\\
\hline
\begin{tabular}[c]{@{}l@{}}~~|  \textbf{try} <s>\textsubscript{1} \textbf{catch} <x>\\~~~~\textbf{then} <s>\textsubscript{2} \textbf{end}\end{tabular} & \multicolumn{2}{c}{Exception context} & \begin{tabular}[c]{@{}l@{}}~~|  \textbf{try} \{ <s>\textsubscript{1} \} \textbf{catch} \{\\~~~~\textbf{case} <pattern> => \{<s>\textsubscript{2}\}\}\end{tabular}\\
~~|  \textbf{raise} <x> \textbf{end} & \multicolumn{2}{c}{Raise exception} & ~~|  \textbf{raise} \{ <x> \}\\
~~|  \{FailedValue <x> <y>\} & \multicolumn{2}{c}{Failed value} & ~~|  failed(<x>, <y>)\\
\hline
~~|  \{NewCell <x> <y>\} & Cell creation & Variable declaration & ~~|  \textbf{var} <x>=<y>\\
~~|  \{Exchange <x> <y> <z>\} & \multicolumn{2}{c}{Cell exchange} & ~~|  exchangeCell(<x>, <y>, <z>)\\
~~|  \{IsDet <x> <y>\} & \multicolumn{2}{c}{Boundness test} & ~~|  isDet(<x>, <y>)\\
\hline
\end{tabular}
\end{adjustbox}
    \caption{The \textit{Oz} and \textit{NewOz} general kernel languages compared}
    \label{tab:OzKernel}
\end{table}


\section{Some translation examples}\label{sec:appendix-examples}
This appendix contains a series of real-world programs written in both \textit{Oz} and \textit{NewOz 2021}, allowing the reader to see the new syntax in action and forge themselves an opinion base on extensive examples.
%! Author = Martin Vandenbussche
\begin{lstlisting}[language=oz,label={lst:lstexamplefibooz},title={Fibonacci : a program using recursion and lambdas (\textit{Oz} version)}]
declare
  Fibo Out Show
in
  Fibo = fun {$ N}
    if N<2 then 1
    else {Fibo X-1}+{Fibo X-2}
    end
  end
  Show = proc {$ S} {Browse S} end
  Out = {Fibo 30}
  {Show Out}
\end{lstlisting}
\begin{lstlisting}[language=newoz,label={lst:lstexamplefibonewoz},title={Fibonacci : a program using recursion and lambdas (\textit{NewOz} version)}]
declare {
  val fibo, out, show
  fibo = def $ (n) {
    if (n<2) {1}
    else {fibo(n-1)+fibo(n-2)}
  }
  show = defproc $ (s) {browse(s)}
  out = fibo(30)
  show(out)
}
\end{lstlisting}
%! Author = Martin Vandenbussche
\begin{lstlisting}[language=oz,label={lst:lstexamplemergeoz},title={Merge sort : working with lists and tail recursion (\textit{Oz} version)}]
declare
  fun {Merge Xs Ys}
    case Xs#Ys
    of nil#Ys then Ys
    [] Xs#nil then Xs
    [] (X|Xr)#(Y|Yr) then
      if X<Y then X|{Merge Xr Ys}
      else Y|{Merge Xs Yr}
      end
    end
  end
  fun {MergeSort Xs}
    fun {MergeSortAcc L1 N}
      if N==0 then nil#L1
      elseif N==1 then (L1.1|nil)#(L1.2)
      elseif N>1 then
        NL=N div 2
        NR=N-NL
        Ys#L2 = {MergeSortAcc L1 NL}
        Zs#L3 = {MergeSortAcc L2 NR}
      in
        {Merge Ys Zs}#L3
      end
    end
  end
in
  {MergeSortAcc Xs {Length Xs}.1
end
\end{lstlisting}
\begin{lstlisting}[language=newoz,label={lst:lstexamplemergenewoz},title={Merge sort : working with lists and tail recursion (\textit{NewOz} version)}]
declare {
  def merge(xs, ys) {
    match (xs#ys) {
      case (nil#ys) => {ys}
      case (xs#nil) => {xs}
      case ((x::xr)#(y::yr)) => {
        if (x<y) {x::merge(xr, ys)}
        else {y::merge(xs, yr)}
      }
    }
  }
  def mergeSort(xs) {
    def mergeSortAcc(l1, n) {
      if (n==0) {nil#l1}
      else if (n==1) {(l1.1::nil)#(l1.2)}
      else if (n>1) {
        val nl = n/2
        val nr = n-nl
        val ys,l2,zs,l3
        ys#l2 = mergeSortAcc(l1, nl)
        zs#l3 = mergeSortAcc(l2, nr)
        merge(ys, zs)#l3
      }
    }
  }
  mergeSortAcc(xs, length(xs)).1
}
\end{lstlisting}

%! Author = Martin Vandenbussche
\begin{lstlisting}[language=oz,label={lst:lstexampleagentsoz},title={Pipeline : a concurrent program example (\textit{Oz} version)}]
declare
  % Consumer logic
  proc {Disp S}
    case S of X|S2 then {Browse X} {Disp S2} end
  end
  % Producer logic
  fun {Prod N} {Delay 1000} N|{Prod N+1} end
  % Transformer logic
  fun {Trans S}
    case S of X|S2 then X*X|{Trans S2} end
  end
  S1 S2
in
  thread S1 = {Prod 1} end % Producer agent
  thread S2 = {Trans S1} end % Producer-Consumer agent
  thread {Disp S2} end % Consumer agent
\end{lstlisting}
\begin{lstlisting}[language=newoz,label={lst:lstexampleagentsnewoz},title={Pipeline : a concurrent program example (\textit{NewOz} version)}]
declare {
  // Consumer logic
  defproc disp(s) {
    match s {
      case x::s2 => {browse(x) disp(s2)}
    }
  }
  // Producer logic
  def prod(n) {
    delay(1000) n::prod(n+1)
  }
  // Transformer logic
  def trans(s) {
    match s {
      case x::s2 => {x*x::trans(s2)}
    }
  }
val s1, s2

thread { s1 = prod(1) } // Producer agent
thread { s2 = trans(s1) } // Producer-Consumer agent
thread { disp(s2) } // Consumer agent
}
\end{lstlisting}

%! Author = Martin Vandenbussche
\begin{lstlisting}[language=oz,label={lst:lstexamplermioz},title={RMI with callback : a message passing program example (\textit{Oz} version)}]
declare
  % Defining a stateless server port object
  fun {NewPortObject2 Proc}
    P S
  in
    {NewPort S P}
    thread
      for M in S do {Proc M} end
    end
    P
  end
  % Server logic
  proc {ServerProc Msg}
    case Msg of calc(X Y Client) then D in
      {Send Client delta(D)}
      Y = X*X+2.0*D*X+D*D+23.0
    end
  end
  Server = {NewPortObject2 ServerProc}
  % Client logic
  proc {ClientProc Msg}
    case Msg of work(Z) then Y in
      {Send Server calc(10.0 Y Client)}
      thread Z=Y+10.0 end
    [] delta(D) then D=0.1
    end
  end
  Client={NewPortObject2 ClientProc}
  % Value to work on
  Z
in
  {Send Client work(Z)}
  {Browse Z}
\end{lstlisting}
\begin{lstlisting}[language=newoz,label={lst:lstexamplerminewoz},title={RMI with callback : a message passing program example (\textit{NewOz} version)}]
declare {
  // Defining a stateless server port object
  def newPortObject2(proc) {
    val p, s
    newPort(s, p)
    thread {
      for (m in s) {proc(m)}
    }
    p
  }
  // Server logic
  defproc serverProc(msg) {
    match msg {
      case 'calc(x, y, client) => {
        val d
        send(client, 'delta(d))
        y = x*x+2.0*d*x+d*d+23.0
      }
    }
  }
  val server = newPortObject2(serverProc)
  // Client logic
  defproc clientProc(msg) {
    match msg {
      case 'work(z) => {
        val y
        send(server, 'calc(10.0, y, client))
        thread {z = y + 10.0}
      }
      case 'delta(d) => { d = 0.1 }
    }
  }
  val client = newPortObject2(clientProc)
  // Value to work on
  val z
  send(client, 'work(z))
  browse(z)
}
\end{lstlisting}

%! Author = Martin Vandenbussche
%! Author = Martin Vandenbussche
\begin{lstlisting}[language=oz,label={lst:lstexampleobjoz},title={Ball playing : a program using active objects (\textit{Oz} version)}]
declare
  fun {NewActive Class Init}
    Obj={New Class Init}
    P
  in
    thread S in
      {NewPort S P}
      for M in S do {Obj M} end
    end
    proc {$ M} {Send P M} end
  end
  class PlayerClass
    attr state others
    meth init(Others)
      state:=0
      others:=Others
    end
    meth ball
      Ran = ({OS.rand} mod {Width @others})+1
    in
      {(@others).Ran ball}
      state:=@state+1
    end
    meth get(X)
      X=@state
    end
  end
  fun {Player Others}
    {NewActive PlayerClass init(Others)}
  end
  P1 P2 P3
in
  % Initialize the game
  P1={Player others(P2 P3)}
  P2={Player others(P1 P3)}
  P3={Player others(P2 P3)}
  % Launch the game
  {P1 ball}
  % Read the state
  local X in {P1 get(X)} {Browse X} end
\end{lstlisting}
\begin{lstlisting}[language=newoz,label={lst:lstexampleobjnewoz},title={Ball playing : a program using active objects (\textit{NewOz} version)}]
declare {
  defproc newActive(class, init, channel) {
    val obj=new(class, init)
    val p
    thread {
      val s
      newPort(s, p)
      for (m in s) { obj.m }
    }
    channel = defproc $(m) { send(p, m) }
  }
  class PlayerClass {
    attr state attr others
    def init(oth) {
      state = 0
      others = oth
    }
    def ball() {
      val ran = (rand() % width(others))+1
      (others.ran).ball()
      state = state+1
    }
    def get(x) {
      x = state
    }
  }
  def player(others) {
    val channel
    newActive(PlayerClass, init(others), channel)
    channel
  }
  val p1, p2, p3
  // Initialize the game
  p1 = player('others(p2, p3))
  p2 = player('others(p1, p3))
  p3 = player('others(p2, p3))
  // Launch the game
  p1(ball())
  // Read the state
  {
    val x
    p1(get(x))
    browse(x)
  }
}
\end{lstlisting}

%! Author = Martin Vandenbussche
\begin{lstlisting}[language=oz,label={lst:lstexampletryoz},title={File operations : an exception-prone program example (\textit{Oz} version)}]
try
  % Some code that could raise exceptions of various nature
  {ProcessFile F}
  1 = 2
catch failure(...) then {Browse 'Caught a failure'}
  [] X then {Browse 'Exception '#X#' when processing file'}
  end
finally {CloseFile F}
end
\end{lstlisting}
\begin{lstlisting}[language=newoz,label={lst:lstexampletrynewoz},title={File operations : an exception-prone program example (\textit{NewOz} version)}]
try {
  // Some code that could raise exceptions of various nature
  processFile(f)
  1 = 2
} catch {
  case 'failure(...) => { browse("Caught a failure") }
  case x => { browse("Exception "+x+" when processing file") }
} finally {
  closeFile(f)
}
\end{lstlisting}

\section{Compilation example}\label{sec:appendix-compilation}
This appendix presents the different states a program's code goes through during the compilation process performed by \texttt{nozc}.
All the following outputs are produced by the command-line utility directly : the usage help is shown as well.
Like briefly mentioned in Section~\ref{sec:ch3-nozc}, \texttt{nozc} provides the option to visualizes the abstract representations of the code at various stages of the compilation process;
this can be very useful in an educational context !
%! Author = Martin Vandenbussche
\begin{lstlisting}[language=newoz,label={lst:compilation-in},title={Input code in \textit{NewOz}}]
declare {
  val a, b=3, c
  def tripleSum(x, y, z) {
    x+y+z
  }
  browse(tripleSum(a,b,c))
}
\end{lstlisting}
\begin{lstlisting}[label={lst:compilation-h},title={Output of the command "\texttt{\$ nozc -h}", showing help regarding the usage of the command}]
(*\texttt{\textcolor{mauve}{nozc 0.0.5-beta}}*)
(c) Martin "Barasingha" Vandenbussche 2021
Run via Picocli 4.6.2-SNAPSHOT
JVM: 16.0.1 (Private Build OpenJDK 64-Bit Server VM 16.0.1+9-Ubuntu-1)
OS: Linux 5.11.0-18-generic amd64
This software is distributed under the BSD license, available at https://github.com/MaVdbussche/nozc/blob/master/LICENSE
(*\texttt{\underline{Usage:}}*)

nozc [(*\texttt{\textcolor{mauve}{-ahpstV}}*)] [(*\texttt{\textcolor{mauve}{--[no-]compile}}*)] [(*\texttt{\textcolor{mauve}{--[no-]keep}}*)] [(*\texttt{\textcolor{mauve}{-d}}*)=<destDirectory>]
     [(*\texttt{\textcolor{mauve}{-o}}*)=<outputFileName>] [(*\texttt{\textcolor{mauve}{-v}}*)=<verbosity>] (*\texttt{\textcolor{mauve}{FILE}}*)...

(*\texttt{\underline{Description:}}*)

Compiles a NewOz program file, or translate it to a valid equivalent Oz program
file.

(*\texttt{\underline{Parameters:}}*)
      (*\texttt{\textcolor{mauve}{FILE}}*)...          The .noz file(s) to compile or translate.

(*\texttt{\underline{Options:}}*)
      (*\texttt{\textcolor{mauve}{--[no-]keep}}*)      Keep the intermediary Oz files in the output folder.
                         True by default
      (*\texttt{\textcolor{mauve}{--[no-]compile}}*)   Continue the program after Oz code generation. The
                         actual Oz code compilation will be executed. True by
                         default
  (*\texttt{\textcolor{mauve}{-t}}*), (*\texttt{\textcolor{mauve}{--tokenize}}*)       Tokenize the NewOz input, print the tokens to STDOUT,
                         and then stop the compilation
  (*\texttt{\textcolor{mauve}{-s}}*), (*\texttt{\textcolor{mauve}{--scan}}*)           Scan/parse the NewOz input, print the AST to STDOUT, and
                         then stop the compilation
  (*\texttt{\textcolor{mauve}{-p}}*), (*\texttt{\textcolor{mauve}{--preAnalyze}}*)     Pre-analyze the NewOz input, print the AST to STDOUT,
                         and then stop the compilation
  (*\texttt{\textcolor{mauve}{-a}}*), (*\texttt{\textcolor{mauve}{--analyze}}*)        Analyze the NewOz input, print the AST to STDOUT, and
                         then stop the compilation
  (*\texttt{\textcolor{mauve}{-v}}*), (*\texttt{\textcolor{mauve}{--verbosity}}*)=<verbosity>
                       The verbosity you want to see in output.
                       Available levels : [OFF, FATAL, ERROR, WARN, INFO,
                         DEBUG, ALL] (default: INFO)
  (*\texttt{\textcolor{mauve}{-o}}*), (*\texttt{\textcolor{mauve}{--out}}*)=<outputFileName>
                       Name of the output file (WITHOUT ANY EXTENSION !). This
                         option will be ignored if you pass more than one input
                         file.
  (*\texttt{\textcolor{mauve}{-d}}*), (*\texttt{\textcolor{mauve}{--directory}}*)=<destDirectory>
                       Output directory for compiled and/or translated files
                         (default: .)
  (*\texttt{\textcolor{mauve}{-h}}*), (*\texttt{\textcolor{mauve}{--help}}*)           Show this help message and exit.
  (*\texttt{\textcolor{mauve}{-V}}*), (*\texttt{\textcolor{mauve}{--version}}*)        Print version information and exit.

(*\texttt{\underline{Exit codes:}}*)
  0   Successful program execution
  1   Exception occurred during program execution
  2   Invalid input (usage)
\end{lstlisting}
\begin{lstlisting}[label={lst:compilation-t},title={Output of the command "\texttt{\$ nozc code.nozc -t}", demonstrating the first part of the compilation (lexer/tokenizer)}]
nozc 0.0.5-beta
(c) Martin "Barasingha" Vandenbussche 2021
Running via Picocli 4.6.2-SNAPSHOT
JVM: 16.0.1 (Private Build OpenJDK 64-Bit Server VM 16.0.1+9-Ubuntu-1)
OS: Linux 5.11.0-18-generic amd64
This software is distributed under the BSD license, available at https://github.com/MaVdbussche/nozc/blob/master/LICENSE
[INFO] : Compiling 1 NewOz file(s) to destination directory "."
[INFO] : Created output file ./code.oz
[INFO] : ==========Scanning started==========
1	 : "declare" = declare
1	 : "{" = {
2	 : "val" = val
2	 : <VARIABLE> = a
2	 : "," = ,
2	 : <VARIABLE> = b
2	 : "=" = =
2	 : <INT> = 3
2	 : "," = ,
2	 : <VARIABLE> = c
3	 : "def" = def
3	 : <VARIABLE> = tripleSum
3	 : "(" = (
3	 : <VARIABLE> = x
3	 : "," = ,
3	 : <VARIABLE> = y
3	 : "," = ,
3	 : <VARIABLE> = z
3	 : ")" = )
3	 : "{" = {
4	 : <VARIABLE> = x
4	 : "+" = +
4	 : <VARIABLE> = y
4	 : "+" = +
4	 : <VARIABLE> = z
5	 : "}" = }
6	 : <VARIABLE> = browse
6	 : "(" = (
6	 : <VARIABLE> = tripleSum
6	 : "(" = (
6	 : <VARIABLE> = a
6	 : "," = ,
6	 : <VARIABLE> = b
6	 : "," = ,
6	 : <VARIABLE> = c
6	 : ")" = )
6	 : ")" = )
7	 : "}" = }
7	 : <EOF> =
[INFO] : ==========Scanning done in 0m:0s:6ms:157µs:827ns==========
\end{lstlisting}
\begin{lstlisting}[label={lst:compilation-s},title={Output of the command "\texttt{\$ nozc code.nozc -s}", demonstrating the second part of the compilation (scanner/syntax analysis)}]
nozc 0.0.5-beta
(c) Martin "Barasingha" Vandenbussche 2021
Running via Picocli 4.6.2-SNAPSHOT
JVM: 16.0.1 (Private Build OpenJDK 64-Bit Server VM 16.0.1+9-Ubuntu-1)
OS: Linux 5.11.0-18-generic amd64
This software is distributed under the BSD license, available at https://github.com/MaVdbussche/nozc/blob/master/LICENSE
[INFO] : Compiling 1 NewOz file(s) to destination directory "."
[INFO] : Created output file ./code.oz
[INFO] : ==========Scanning started==========
[INFO] : ==========Scanning done in 0m:0s:2ms:481µs:399ns==========
[INFO] : ==========Parsing input==========
<InteractiveStatement line="1">
<StatementBlock line="1">
  <ConstantDeclaration line="2" name="a">
    <Value : none>
  </ConstantDeclaration>
  <ConstantDeclaration line="2" name="c">
    <Value : none>
  </ConstantDeclaration>
  <ConstantDeclaration line="2" name="b">
    <Value :>
    <Literal image="3">
  </ConstantDeclaration>
  <FunctionDeclaration line="3" name="tripleSum">
    <Argument>
      <Variable>
        <name:x constant:true>
      </Variable>
    </Argument>
    <Argument>
      <Variable>
        <name:y constant:true>
      </Variable>
    </Argument>
    <Argument>
      <Variable>
        <name:z constant:true>
      </Variable>
    </Argument>
    <ExpressionBlock line="3">
      <BinaryExpression line="4" type="" operator="+">
        <Lhs>
          <BinaryExpression line="4" type="" operator="+">
            <Lhs>
              <Variable>
                <name:x constant:true>
              </Variable>
            </Lhs>
            <Rhs>
              <Variable>
                <name:y constant:true>
              </Variable>
            </Rhs>
          </BinaryExpression>
        </Lhs>
        <Rhs>
          <Variable>
            <name:z constant:true>
          </Variable>
        </Rhs>
      </BinaryExpression>
    </ExpressionBlock>
  </FunctionDeclaration>
  <CallProcedureStatement line="6" name="browse">
    <Arguments>
      <Argument>
        <CallFunctionExpression line="6" name="tripleSum">
          <Arguments>
            <Argument>
              <Variable>
                <name:a constant:true>
              </Variable>
            </Argument>
            <Argument>
              <Variable>
                <name:b constant:true>
              </Variable>
            </Argument>
            <Argument>
              <Variable>
                <name:c constant:true>
              </Variable>
            </Argument>
          </Arguments>
        </CallFunctionExpression>
      </Argument>
    </Arguments>
  </CallProcedureStatement>
</StatementBlock>
</InteractiveStatement>
[INFO] : ==========Parsing done in 0m:0s:25ms:661µs:614ns==========
\end{lstlisting}
\begin{lstlisting}[label={lst:compilation-a},title={Output of the command "\texttt{\$ nozc code.nozc -a}", demonstrating the third part of the compilation (semantic analysis)}]
nozc 0.0.5-beta
(c) Martin "Barasingha" Vandenbussche 2021
Running via Picocli 4.6.2-SNAPSHOT
JVM: 16.0.1 (Private Build OpenJDK 64-Bit Server VM 16.0.1+9-Ubuntu-1)
OS: Linux 5.11.0-18-generic amd64
This software is distributed under the BSD license, available at https://github.com/MaVdbussche/nozc/blob/master/LICENSE
[INFO] : Compiling 1 NewOz file(s) to destination directory "."
[INFO] : Created output file ./code.oz
[INFO] : ==========Scanning started==========
[INFO] : ==========Scanning done in 0m:0s:1ms:627µs:381ns==========
[INFO] : ==========Parsing input==========
[INFO] : ==========Parsing done in 0m:0s:14ms:894µs:931ns==========
[INFO] : ==========Pre-analyzing input==========
[INFO] : ==========Pre-analyzing done in 0m:0s:9ms:49µs:596ns==========
[INFO] : ==========Analyzing input==========
<InteractiveStatement line="1">
<StatementBlock line="1">
  <ConstantDeclaration line="2" name="a">
    <Value : none>
  </ConstantDeclaration>
  <ConstantDeclaration line="2" name="c">
    <Value : none>
  </ConstantDeclaration>
  <ConstantDeclaration line="2" name="b">
    <Value :>
    <Literal image="3">
  </ConstantDeclaration>
  <FunctionDeclaration line="3" name="tripleSum">
    <Argument>
      <Variable>
        <name:x constant:true>
      </Variable>
    </Argument>
    <Argument>
      <Variable>
        <name:y constant:true>
      </Variable>
    </Argument>
    <Argument>
      <Variable>
        <name:z constant:true>
      </Variable>
    </Argument>
    <ExpressionBlock line="3">
      <BinaryExpression line="4" type="Any" operator="+">
        <Lhs>
          <BinaryExpression line="4" type="Any" operator="+">
            <Lhs>
              <Variable>
                <name:x constant:true>
              </Variable>
            </Lhs>
            <Rhs>
              <Variable>
                <name:y constant:true>
              </Variable>
            </Rhs>
          </BinaryExpression>
        </Lhs>
        <Rhs>
          <Variable>
            <name:z constant:true>
          </Variable>
        </Rhs>
      </BinaryExpression>
    </ExpressionBlock>
  </FunctionDeclaration>
  <CallProcedureStatement line="6" name="browse">
    <Arguments>
      <Argument>
        <CallFunctionExpression line="6" name="tripleSum">
          <Arguments>
            <Argument>
              <Variable>
                <name:a constant:true>
              </Variable>
            </Argument>
            <Argument>
              <Variable>
                <name:b constant:true>
              </Variable>
            </Argument>
            <Argument>
              <Variable>
                <name:c constant:true>
              </Variable>
            </Argument>
          </Arguments>
        </CallFunctionExpression>
      </Argument>
    </Arguments>
  </CallProcedureStatement>
</StatementBlock>
</InteractiveStatement>
[INFO] : ==========Analyzing done in 0m:0s:24ms:247µs:102ns==========
\end{lstlisting}
\begin{lstlisting}[label={lst:compilation-a},title={Resulting \textit{Oz} code, demonstrating the successful compilation process}]
declare
A
B=3
C
fun{TripleSum X Y Z}
X + Y + Z
end
in
{Browse {TripleSum A B C} }
\end{lstlisting}
\textit{The formatting of the output \textit{Oz} code, as we mentioned before, should be one of the main attention points in future improvements of \texttt{nozc}.}


\section{Documentation and tutorial}\label{sec:appendix-doc}
The interested reader will find tutorials and documentation about the \textit{NewOz} syntax and the usage of its compiler on the \texttt{nozc} \textit{GitHub} repository~\cite{NozcGitHub}.
This documentation, like the rest of this repository, is completely open source and anyone is welcome to collaborate on its writing.
Our long-term goal is for this knowledge base to serve as a complement to a future version of the \textit{Concepts, techniques, and models of computer programming} book, which would rely on \textit{NewOz} instead of \textit{Oz}.
We chose not to include them here because it added little value to this paper, since they are publicly available online.
Furthermore, they will most likely evolve in the future with the next releases of \texttt{nozc} and the upcoming iterations of \textit{NewOz}, making their version included here obsolete.