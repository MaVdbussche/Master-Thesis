%! Author = Martin Vandenbussche

\appendix
\chapter*{Appendices}
\addcontentsline{toc}{chapter}{Appendices}
\renewcommand{\thesection}{\Alph{section}}

\section{Appendix A : \textit{NewOz} EBNF Grammar (2021 version)}\label{sec:appendix-grammar}
This EBNF grammar is a reworked version of the one provided in the appendices of Jean-Pacifique Mbonyincungu's thesis,
removing left-recursion problems and including changes made in the syntax since then.\newline
This EBNF grammar is thus suitable for recursive descent, and is the grammar used by \texttt{nozc}.
\begin{lstlisting}[label={lst:newOzEBNF},frame=single,basicstyle=\footnotesize\ttfamily,escapeinside={(*}{*)}]
EBNF grammar for newOz, suitable for recursive descent
- Note that the concatenation symbol in EBNF (comma) is
omitted for readability reasons
Notation      Meaning
===========================================================================
(*\epsilon*)             singleton containing the empty word
(*$(w)$*)           grouping of regular expressions
(*$[w]$*)           union of \epsilon with the set of words w (optional group)
(*$\{w\}$*)          zero or more times w
(*$\{w\}+$*)         one or more times w
(*$w_1~w_2$*)         concatenation of (*$w_1$*) with (*$w_2$*)
(*$w_1 | w_2$*)         logical union of (*$w_1$*) and (*$w_2$*) (OR)
(*$w_1-w_2$*)       difference of (*$w_1$*) and (*$w_2$*)

// Interactive statements [ENTRYPOINT]
interStatement ::= statement
            | DECLARE LCURLY {declarationPart}+ [interStatement] RCURLY

statement ::= nestConStatement
            | nestDecVariable
            | SKIP
            | SEMI
            //| DECLARE statement //TODO removed bcs matched in interStatement ?
            | RETURN expression

expression ::= nestConExpression
            | nestDecAnonym
            | DOLLAR
            | term
            | THIS
            | LCURLY expression {expression} RCURLY //TODO not implemented like this

parExpression ::= LPAREN expression RPAREN

inStatement ::= LCURLY {declarationPart} {statement} RCURLY //TODO added possibility for multiple chained statements
            | LCURLY {declarationPart} expression RCURLY

inExpression ::= LCURLY {declarationPart} [statement] expression RCURLY
            | LCURLY {declarationPart} statement RCURLY

nestConStatement ::= assignmentExpression
            | variable LPAREN {expression {COMMA expression}} RPAREN
            | {LCURLY}+ expression {expression} {RCURLY}+
            | LPAREN inStatement RPAREN
            | IF parExpression inStatement
                {ELSE IF LPAREN expression RPAREN inStatement}
                [ELSE inStatement]
            | MATCH expression LCURLY
                {CASE caseStatementClause}+
                [ELSE inStatement]
              RCURLY
            | FOR LPAREN {loopDec}+ RPAREN inStatement
            | TRY inStatement
                [CATCH LCURLY
                    {CASE caseStatementClause}+
                RCURLY]
                [FINALLY inStatement]
            | RAISE inExpression
            | THREAD inStatement
            | LOCK [LPAREN expression RPAREN] inStatement

nestConExpression ::= LPAREN expression RPAREN
            | variable LPAREN {expression {COMMA expression}} RPAREN
            | IF LPAREN expression RPAREN inExpression
                {ELSE IF LPAREN expression RPAREN inExpression}
                [ELSE inExpression]
            | MATCH expression LCURLY
                {CASE caseExpressionClause}+
                [ELSE inExpression]
              RCURLY
            | FOR LPAREN {loopDec}+ RPAREN inExpression
            | TRY inExpression
                [CATCH LCURLY
                    {CASE caseExpressionClause}+
                RCURLY]
                [FINALLY inStatement]
            | RAISE inExpression
            | THREAD inExpression
            | LOCK [LPAREN expression RPAREN] inExpression

nestDecVariable ::= DEFPROC variable
                LPAREN {pattern {COMMA pattern}} RPAREN inStatement
            | DEF [LAZY] variable
                LPAREN {pattern {COMMA pattern}} RPAREN inExpression
            | FUNCTOR [variable] {
                (IMPORT importClause {COMMA importClause}+)
                    | (EXPORT exportClause {COMMA exportClause}+)
                }
                inStatement
            | CLASS variableStrict [classDescriptor] LCURLY
                {classElementDef} RCURLY

nestDecAnonym ::= DEFPROC DOLLAR
                LPAREN {pattern {COMMA pattern}} RPAREN inStatement
            | DEF [LAZY] DOLLAR
                LPAREN {pattern {COMMA pattern}} RPAREN inExpression
            | FUNCTOR [DOLLAR] {
                (IMPORT importClause {COMMA importClause}+)
                    | (EXPORT exportClause {COMMA exportClause}+)
                }
              inStatement
            | CLASS DOLLAR [classDescriptor] LCURLY
                {classElementDef} RCURLY

importClause ::=  variable
                [LPAREN (atom|int)[COLON variable]
                {COMMA (atom|int)[COLON variable]} RPAREN]
                [FROM atom]

exportClause ::= [(atom|int) COLON] variable

classElementDef ::= DEF methHead [ASSIGN variable]
                (inExpression|inStatement)
            | classDescriptor

caseStatementClause ::= pattern {(LAND|LOR) conditionalExpression}
                IMPL inStatement

caseExpressionClause ::= pattern {(LAND|LOR) conditionalExpression}
                IMPL inExpression

assignmentExpression ::= conditionalExpression
                [(ASSIGN|PLUSASS|MINUSASS|DEFINE) assignmentExpression]

conditionalExpression ::= conditionalOrExpression

conditionalOrExpression ::= conditionalAndExpression
                {LOR conditionalAndExpression}

conditionalAndExpression ::= equalityExpression
                {LAND equalityExpression}

equalityExpression ::= relationalExpression
                {EQUAL relationalExpression}

relationalExpression ::= additiveExpression
                [(GT|GE|LT|LE) additiveExpression]

additiveExpression ::= multiplicativeExpression
                {(PLUS|MINUS) multiplicativeExpression}

multiplicativeExpression ::= unaryExpression
                {(STAR|SLASH|MODULO) unaryExpression}

unaryExpression ::= (INC|DEC|MINUS|PLUS) unaryExpression
            | simpleUnaryExpression

simpleUnaryExpression ::= LNOT unaryExpression
            | postfixExpression

postfixExpression ::= primary {selector} {(DEC|INC)}

primary ::= parExpression
            | THIS DOT
                variable [LPAREN {expression {COMMA expresssion}} RPAREN]
            | SUPER LPAREN variableStrict RPAREN DOT
                variable [LPAREN {expression {COMMA expression}} RPAREN]
            | literal
            | qualifiedIdentifier
            | initializer

// Terms and patterns
term ::= atom
            | atomLisp LPAREN
                [[feature COLON] expression
                {COMMA [feature COLON] expression}] RPAREN

pattern ::= {LNOT} variable | int | float | character | atom | string
            | UNIT | TRUE | FALSE | UNDERSCORE | NIL //TODO we can remove character bcs of int representation ?
            | atomLisp LPAREN [[feature COLON] pattern
                {COMMA [feature COLON] pattern} [COMMA ELLIPSIS]] RPAREN
            | LPAREN pattern {(HASHTAG|COLCOL) pattern} RPAREN
            | LBRACK [pattern {COMMA pattern}] RBRACK
            | LPAREN pattern RPAREN

declarationPart ::= (VAL|VAR) (variable|pattern)
                ASSIGN (expression|statement)
                {COMMA (variable|pattern) ASSIGN (expression|statement)} //TODO why statement ?

loopDec ::= variable IN expression [DOTDOT expression] [SEMI expression]
            | variable IN expression SEMI expression SEMI expression
            | BREAK COLON variable
            | CONTINUE COLON variable
            | RETURN COLON variable
            | DEFLT COLON expression
            | COLLECT COLON variable

literal ::= TRUE | FALSE | NIL | int | string | character | float //TODO we can remove character bcs of int representation ?

//label ::= UNIT | TRUE | FALSE | variable | atom //TODO actually not used anywhere

feature ::= UNIT | TRUE | FALSE | atom | int | NIL //TODO not implemented like this

classDescription ::= EXTENDS variableStrict {COMMA variableStrict}+
            | ATTR variable [ASSIGN expression]
            | PROP variable

//attrInit ::= ([LNOT] variable | atom | UNIT | TRUE | FALSE) [COLON expression] //TODO not implemented

methHead ::= ([LNOT] variableStrict | atomLisp | UNIT | TRUE | FALSE) //TODO not implemented like this
                [LPAREN methArg {COMMA methArg}
                [COMMA ELLIPSIS] [DOLLAR] RPAREN]

methArg ::= [feature COLON] (variable | UNDERSCORE) [LE expression]

variableStrict ::= UPPERCASE {ALPHANUM}
              | LACCENT {VARIABLECHAR | PSEUDOCHAR} LACCENT

variable ::= LOWERCASE {ALPHANUM}
          | APOSTROPHE {VARIABLECHAR | PSEUDOCHAR} APOSTROPHE //TODO really ?

atom ::= atomLisp
        | RACCENT {ATOMCHAR | PSEUDOCHAR} RACCENT

atomLisp ::= APOSTROPHE (LOWERCASE | UPPERCASE) {ALPHANUM}

string ::= QUOTE {STRINGCHAR | PSEUDOCHAR} QUOTE

character ::= CHARINT
            | DEGREE CHARCHAR
            | DEGREE PSEUDOCHAR
            | CHAR // TODO in this case we should send a warning during analysis that it is not supported in underlying oldOz

int ::= [MINUS] DIGIT
        | [MINUS] NONZERODIGIT {DIGIT}
        | [MINUS] "0" {OCTDIGIT}+
        | [MINUS] ("0x"|"0X") {HEXDIGIT}+
        | [MINUS] ("0b"|"0B") {BINDIGIT}+

float ::= [MINUS] {DIGIT}+ DOT {DIGIT} [("e" | "E")[~]{DIGIT}+]

boolean ::= TRUE | FALSE

\end{lstlisting}


\section{Appendix B : Lexical Grammar (2021 version)}\label{sec:appendix-lexical-grammar}
This is the lexical grammar used by \texttt{nozc}.
%! Author = Martin Vandenbussche
\begin{lstlisting}[label={lst:newOzLexical},language=ebnf]
Notation      Meaning
===========================================================================
;\epsilon;             singleton containing the empty word
;$(w)$;           grouping of regular expressions
;$[w]$;            union of ;\epsilon; with the set of words ;$w$; (optional group)
;$\{w\}$;           zero or more times ;$w$;
;$\{w\}+$;         one or more times ;$w$;
;$w_1~w_2$;         concatenation of ;$w_1$; with ;$w_2$;
;$w_1 | w_2$;         logical union of ;$w_1$; and ;$w_2$; (OR)
;$w_1-w_2$;       difference of ;$w_1$; and ;$w_2$;

// White spaces - ignored
WHITESPACE ::= (" "|"\b"|"\t"|"\n"|"\r"|"\f")

// Comments - ignored
("//" {~("\n"|"\r")} ("\n"|"\r"|"\r\n")
// Multi-line comments - ignored
"/*" {CHAR - "*/"} "*/"

// Reserved keywords
AT      ::= "at"
ATTR    ::= "attr"
BREAK   ::= "break"
CASE    ::= "case"
CATCH   ::= "catch"
CLASS   ::= "class"
CONTINUE::= "continue"
DECLARE ::= "declare"
DEF     ::= "def"
DEFPROC ::= "defproc"
DO      ::= "do"
ELSE    ::= "else"
EXPORT  ::= "export"
EXTENDS ::= "extends"
FALSE   ::= "false"
FINALLY ::= "finally"
FOR     ::= "for"
FROM    ::= "from"
FUNCTOR ::= "functor"
IF      ::= "if"
IMPORT  ::= "import"
IN      ::= "in"
LAZY    ::= "lazy"
LOCK    ::= "lock"
MATCH   ::= "match"
NIL     ::= "nil"
OR      ::= "or"
PROP    ::= "prop"
RAISE   ::= "raise"
RETURN  ::= "return"
SKIP    ::= "skip"
SUPER   ::= "super"
THIS    ::= "this"
THREAD  ::= "thread"
TRUE    ::= "true"
TRY     ::= "try"
UNIT    ::= "unit"
VAL     ::= "val"
VAR     ::= "var"

// Operators
ASSIGN      ::= "="
PLUSASS     ::= "+="
MINUSASS    ::= "-="
EQUAL       ::= "=="
NE          ::= "\\="
LT          ::= "<"
GT          ::= ">"
LE          ::= "<="
GE          ::= ">="
IMPL        ::= "=>"
LAND        ::= "&&"
LOR         ::= "||"
LNOT        ::= "!"
MINUS       ::= "-"
PLUS        ::= "+"
STAR        ::= "*"
SLASH       ::= "/"
MODULO      ::= "%"
HASHTAG     ::= "#"
UNDERSCORE  ::= "_"
DOLLAR      ::= "$"
APOSTROPHE  ::= "'"
QUOTE       ::= "\""
DEGREE      ::= "°"
COLCOL      ::= "::"
COMMA       ::= ","
LBRACK      ::= "["
LCURLY      ::= "{"
LPAREN      ::= "("
RBRACK      ::= "]"
RCURLY      ::= "}"
RPAREN      ::= ")"
SEMI        ::= ";;"
COLON       ::= ":"
DOT         ::= "."
DOTDOT      ::= ".."
ELLIPSIS    ::= "..."

// Literals
VARIABLESTRICT ::= UPPERCASE{ALPHANUM}
                    | "`"(ESC | PSEUDO_CHAR | ~("`"|"\\"|"\n"|"\r") )"`")
VARIABLE       ::= LOWERCASE{ALPHANUM}
ATOM           ::= (ATOMLISP | "´" (ESC | PSEUDO_CHAR | ~("\\"|"\n"|"\r") ) "´")
ATOMLISP       ::= "'" {LETTER}
STRING         ::= "\"" { ESC | PSEUDO_CHAR | ~("\""|"\\"|"\n"|"\r") } "\""
CHARACTER      ::= (DEGREE(CHARCHAR | PSEUDO_CHAR)
                    | "'" (ESC | ~("'"|"\\"|"\n"|"\r") ) "'" )
INT            ::= (DECINT | HEXINT | OCTINT | BININT)
FLOAT          ::= {DIGIT}+ DOT {DIGIT} [ ("e"|"E")["~"]{DIGIT}+ ]
UPPERCASE      ::= "A"|...|"Z"
LOWERCASE      ::= "a"|...|"z"
LETTER         ::= "A"|...|"Z"|"a"|...|"z"
DIGIT          ::= "0"|...|"9"
NON_ZERO_DIGIT ::= "1"|...|"9"
CHARINT        ::= ("0"|...|"9") | ("1"|...|"9")("0"|...|"9")
                    | "1"("0"|...|"9")("0"|...|"9")
                    | "2"("0"|...|"4")("0"|...|"9")|"25"("0"|...|"5") // (0-255)
ALPHANUM       ::= (UPPERCASE | LOWERCASE | DIGIT | UNDERSCORE)
DECINT         ::= ("0" | (NON_ZERO_DIGIT{DIGIT}))
HEXINT         ::= "0" ("x"|"X") {HEXDIGIT}+
OCTINT         ::= "0" {OCTDIGIT}+
BININT         ::= "0" ("b"|"B") {BINDIGIT}+
OCTDIGIT       ::= "0"|...|"7"
HEXDIGIT       ::= (DIGIT | ("A"|...|"F") | ("a"|...|"f"))
BINDIGIT       ::= ("0"|"1")
ESC            ::= "\\" ESCAPE_CHAR
ESCAPE_CHAR    ::= ("a"|"b"|"f"|"n"|"r"|"t"|"\\"|"'"|"\""|DEGREE)
CHARCHAR       ::= ~("\\")
// In the classes of words <variable>, <atom>, <string>, and <character>, we use pseudo-characters, which represent single characters in different notations.
PSEUDO_CHAR    ::= ( "\\"(OCTDIGIT)(OCTDIGIT)(OCTDIGIT) ) | ( "\\"("x"|"X")(HEXDIGIT)(HEXDIGIT) )
// End of file
EOF            ::= "<end of file>"

\end{lstlisting}

\section{Appendix C : Kernel language}\label{sec:appendix-kernel}
The following table~\ref{tab:OzKernel} gives the kernel language of the \textit{general computational model} of \textit{Oz}, as defined in~\cite{van2004concepts}.
The right-most column gives the \textit{NewOz} equivalent, which allows the user to quickly compare both approaches.
%! Author = Martin Vandenbussche
\begin{table}[H]
\begin{adjustbox}{width=1.2\textwidth,center}
\begin{tabular}{|lr|ll|}
\hline
<s> ::= & & & <s> ::=\\
~~~~\textbf{skip} & \multicolumn{2}{c}{Empty statement} & ~~~~\textbf{skip}\\
~~|  <s>\textsubscript{1} <s>\textsubscript{2} & \multicolumn{2}{c}{Statement sequence} & ~~|  <s>\textsubscript{1} <s>\textsubscript{2}\\
~~|  \textbf{local} <x> \textbf{in} <s> \textbf{end} & Variable creation & Value declaration & ~~|  \{ \textbf{val} <x> <s> \}\\
~~|  <x>\textsubscript{1}=<x>\textsubscript{2} & Variable binding & Variable/value binding & ~~|  <x>\textsubscript{1}=<x>\textsubscript{2}\\
~~|  <x>=<v> & Value creation & Variable/Value assignment & ~~|  <x>=<v>\\
\hline
~~|  \{<x> <y>\textsubscript{1} \ldots <y>\textsubscript{n}\} & \multicolumn{2}{c}{Procedure application} & ~~|  x(<y>\textsubscript{1}, \ldots, <y>\textsubscript{n})\\
\begin{tabular}[c]{@{}l@{}}~~|  \textbf{if} <x> \textbf{then} <s>\textsubscript{1}\\~~~~\textbf{else} <s>\textsubscript{2} \textbf{end}\end{tabular} & \multicolumn{2}{c}{Conditional} & \begin{tabular}[c]{@{}l@{}}~~|  \textbf{if} (<x>) \{ <s>\textsubscript{1} \}\\~~~~\textbf{else} \{ <s>\textsubscript{2} \}\end{tabular}\\
\begin{tabular}[c]{@{}l@{}}~~|  \textbf{case} <x> \textbf{of} <pattern>\\~~~~\textbf{then} <s>\textsubscript{1}\\~~~~\textbf{else} <s>\textsubscript{2} \textbf{end}\end{tabular} & \multicolumn{2}{c}{Pattern matching} & \begin{tabular}[c]{@{}l@{}}~~|  \textbf{match} <x> \{\\~~~~\textbf{case} <pattern> => \{<s>\textsubscript{1}\}\\~~~~\textbf{else} \{ <s>\textsubscript{2} \} \}\end{tabular}\\
~~|  \textbf{thread} <s> \textbf{end} & \multicolumn{2}{c}{Thread creation} & ~~|  \textbf{thread} \{ <s> \}\\
~~|  \{ByNeed <x> <y>\} & \multicolumn{2}{c}{Trigger creation} & ~~|  byNeed(<x>, <y>)\\
\hline
~~|  \{NewName <x>\} & \multicolumn{2}{c}{Name creation} & ~~| newName(<x>)\\
~~|  <y>=!!<x> & \multicolumn{2}{c}{Read-only view} & \emph{Not implemented yet}\\
\hline
\begin{tabular}[c]{@{}l@{}}~~|  \textbf{try} <s>\textsubscript{1} \textbf{catch} <x>\\~~~~\textbf{then} <s>\textsubscript{2} \textbf{end}\end{tabular} & \multicolumn{2}{c}{Exception context} & \begin{tabular}[c]{@{}l@{}}~~|  \textbf{try} \{ <s>\textsubscript{1} \} \textbf{catch} \{\\~~~~\textbf{case} <pattern> => \{<s>\textsubscript{2}\}\}\end{tabular}\\
~~|  \textbf{raise} <x> \textbf{end} & \multicolumn{2}{c}{Raise exception} & ~~|  raise \{ <x> \}\\
~~|  \{FailedValue <x> <y>\} & \multicolumn{2}{c}{Failed value} & ~~|  failed(<x>, <y>)\\
\hline
~~|  \{NewCell <x> <y>\} & Cell creation & Variable declaration & ~~|  var <x>=<y>\\
~~|  \{Exchange <x> <y> <z>\} & \multicolumn{2}{c}{Cell exchange} & ~~|  exchangeCell(<x>, <y>, <z>)\\
~~|  \{IsDet <x> <y>\} & \multicolumn{2}{c}{Boundness test} & ~~|  isDet(<x>, <y>)\\
\hline
\end{tabular}
\end{adjustbox}
    \caption{The \textit{Oz} and \textit{NewOz} general kernel languages compared}
    \label{tab:OzKernel}
\end{table}


\section{Appendix D : Some Examples}\label{sec:appendix-examples}
Code examples : Oz vs \textit{NewOz}
%! Author = Martin Vandenbussche
\begin{itemize}
    \item everytime : show Oz + NewOz2020 + NewOz2021
    \item lambdas (functions and procedure) - fibo example
    \item classic small math stuff with ifs
    \item OOP
    \item tail-recursion and lists/streams syntax
\end{itemize}


\section{Appendix E : Compilation example}\label{sec:appendix-compilation}

\section{Appendix F : Documentation and tutorial}\label{sec:appendix-doc}
Move this as the first appendix ?