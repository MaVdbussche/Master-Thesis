%! Author = Martin Vandenbussche

Extract from Jean-Pacifique's thesis :
One of the key elements of this project is that compatibility has to be maintained with the existing Mozart system, for the official release of Mozart2.
The idea of writing a new compiler has thus quickly been set aside, as it would drastically increase the time and complexity requirements of the project.
Instead, the previous thesis brought forward the idea of writing a syntax parser, that would serve as a compatibility layer between the so-called "newOz" syntax, and the existing Oz syntax.\cite{jpthesis}
NewOz code will be translated, and then fed to the existing Oz compiler.\newline
This approach has been selected because it is easier to implement, relies on more modern technologies than the existing Mozart compiler, and will allow for more flexibility down the line because of the nature of the underlying Scala code of the Parser.
We should however keep in mind the major limitation that this approach brings : error messages generated by the Mozart compiler will be way harder to interpret by the end-user.
One of the goals of the project being that this Parser-to-compiler behavior stays transparent to the programmer, we should keep in mind that future Oz learners will know nothing of the current Oz syntax.
As such, the compiler output will probably be obscure to them, and it is probably a good idea to try to alleviate this confusion as much as possible.\newline

IMPLEMENTATION - PAN TECHNIQUE/PRATIQUE

\section{The need for something else}\label{sec:ch2WhyCompiler}
Why did we feel like we needed to make a full compiler, and why we didn't go all the way to the machine language

\section{A quick introduction to compilers}\label{sec:ch2WhatIsCompiler}
How do compilers work in general ?
Try to keep it not TOO technical and long.

\section{Nozc dissected}\label{sec:ch2HowCompiler}
General description of the inner workings of the compiler.
Do not go in ridiculous details, as the code is well documented and available.
Use an example and show its evolution when going through the compiler.

\section{Technologies used}\label{sec:ch2TechStack}
Why Java (pros and cons) (mention Java here for 1st time -> forces me to be generic in previous section). picocli, JavaCC.\newline
Try to make it short

\section{Advantages of this approach}\label{sec:ch2CompilerPros}
Persuade the reader that it is a great platform to reuse and build upon.
How useful it can be for future works on the syntax

\section{Limitations of this approach}\label{sec:ch2CompilerCons}
Let's keep it real, it is not perfect either\ldots