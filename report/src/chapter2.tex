%! Author = Martin Vandenbussche

In this chapter, we will describe the general objectives we felt were important to attain with the \textit{NewOz} syntax, as well as the characteristics that we deemed desirable for this syntax to have.
We will then review the important changes that were made  with respects to \textit{NewOz 2020}, and explain the motivation behind said changes.
The goal here is not to repeat what was said before by M. Mbonyincungu in~\cite{jpthesis};
the interested reader can consult his thesis for a systematic review of the syntactic changes proposed last year.
We will instead focus on syntax elements that were either overlooked in that thesis, or that have been significantly modified during this year's work.
Finally, we will conclude the chapter by evaluating whether this new version of \textit{NewOz} fulfills its announced objectives, and outline potential improvements areas that we identified at that stage of the work.\newline

\section{Our purpose : the big picture}\label{sec:ch2-goal}
The main goal of the multi-year project, as we have said before, is to create a new syntax that feels more modern to new programmers than the existing one, while keeping in the language all the functionalities that \textit{Oz} currently has.
Furthermore, this syntax should be able to integrate new concepts and paradigms in the future, in a way that is consistent with existing language features.
In his thesis, M. Mbonyincungu decided to [verb] the design process around \textit{Scala} and \textit{Ozma}, while incorporating some elements from other languages in limited places.
This has the main advantage of making the syntax very consistent from the start, provided the design process [pays attention] to only introduce elements from other languages when necessary;
at any given moment, one has to ask themselves if the value provided by this new, foreign element is worth the inevitable inconsistency it will cause in the syntax, or in the general philosophy of the language.\newline

In that regard, we feel like \textit{NewOz 2020} has been successful : this new syntax feels modern and more in par with the syntax's of languages used nowadays, but it also feels more consistent than \textit{Oz} in some places.
Object-oriented syntax, in particular, underwent some major changes that make it way more pleasing to use.
But as M. Mbonyincungu mentioned himself, \textit{NewOz 2020} still needed maturation : it is a huge step in the right direction, but it still has flaws that need to be fixed before it could be used by online programmers or as a teaching tool.
In the next section, we will go over some of those changes that we feel are worth mentioning, because they raised interesting questions and reflections;
the reader will find extensive code examples covering those changes in appendix~\ref{sec:appendix-examples}, in the form of programs written in \textit{Oz}, \textit{NewOz 2020} and \textit{NewOz 2021} presented side by side.

\section{In practice : a review of the relevant syntax elements}\label{sec:ch2-review}
As mentioned above, we will not cover the thought process behind every syntactic element in detail, as it would be a repetition of the work done in M. Mbonyincungu's thesis.
However, we still feel it is important to briefly describe them, in order to provide a global overview of \textit{NewOz 2021} in a single place.\newline
You will thus find some small code examples targeting a specific syntactic element;
the interested reader can find complete program examples in appendix~\ref{sec:appendix-examples}.\newline

In this section, code examples will present \textit{Scala} and \textit{Ozma} as a single entity, since, as we presented before, \textit{Ozma} is an extension of \textit{Scala}'s syntax;
on the other hand, \textit{NewOz 2020} and \textit{2021} snippets, when applicable, will be separated to emphasize their differences and highlight the specific contributions made in this year's version.

\subsection{Variables and values}
A first syntax element we reviewed in \textit{NewOz 2020} was the declaration and use of variables and values.
While the introduction of keywords \texttt{var} and \texttt{val} is a big improvement, and a great way to hide the behaviour of cells in Oz, the possibility that was introduced to write a semicolon ";" at the end of a line declaring variables immediately caught our attention.
To quote M. Mbonyincungu's thesis, "the ";" end of line token is just a random addition inspired from \textit{Scala} to allow those with \textit{Scala} creating an unbound value with a peace of mind" (\textit{sic}).
This justification seems to us precarious at best;
not only does it go again the general idea in \textit{Oz} that carriage returns are the preferred way to delimit statements, but it also is the only use of the semicolon character in the whole syntax.
We felt like two options were available : either use this delimiter for every statement in the syntax, like in Java for example, or never use it at all.
We decided to go for the second option, if only because it stays closer to the original \textit{Oz} philosophy.\newline

Another idea that was left out in \textit{NewOz 2021} was the support for variables in both uppercase and lowercase;
this idea seems problematic to us because it goes against the conventions used by most programming languages.
Not keeping this in the new syntax also allows us to save capitalized nouns for class names (see later).\newline

Cells in \textit{Oz} provide a specific syntax for reading and writing their content, using respectively the tokens \texttt{@} and \texttt{:=}, whereas variables use the \texttt{=} sign.
\textit{NewOz 2020} proposed to keep this syntax for the now-called \texttt{var}s, arguing that it allows to better showcase the fundamental difference between cells and variables in \textit{Oz}.
Our take is that using the more intuitive \texttt{=} token in both places is not only aesthetically more pleasing than the dated \texttt{@} and \texttt{:=} symbols, but it also doesn't take away the teaching opportunity that \textit{Oz}'s immutable variables represent.
Indeed, the unification of the notation allows new programmers, that haven't used \textit{Oz} in the past, to use \texttt{var}s and \texttt{val}s in an intuitive manner, with the resulting behaviour that they expect;
on the other hand, students using \textit{NewOz} can receive an explanation of the reason why \texttt{var}s are mutable, and how this is in fact implemented in \textit{Oz} and its kernel language.
For those reasons, we felt like using the more standard \texttt{=} token everywhere was a preferable solution in this case.\newline
A comparison of the successive versions of this syntax can be found in Figure~\ref{fig:code-comp-var}.
%\newline
\begin{figure}
    \noindent\begin{minipage}{.49\textwidth}
    \begin{lstlisting}[title={Oz},language=oz]
local
  X
  Y = 8
  Z = {NewCell 5}
in
  Z := Y
end
    \end{lstlisting}
\end{minipage}
\hfill
    \noindent\begin{minipage}{.49\textwidth}
    \begin{lstlisting}[title={Scala/Ozma},language=scala]
... {
  val x: Int
  val y = 8
  var z: Int = 5

  z = y
}
    \end{lstlisting}
\end{minipage}
%\newline
%\noindent
    \noindent\begin{minipage}{.49\textwidth}
    \begin{lstlisting}[title={NewOz 2020},language=newoz]
... {
  val x;
  val Y = 8
  var z = 5
  z := Y
}
    \end{lstlisting}
\end{minipage}
\hfill
    \noindent\begin{minipage}{.49\textwidth}
    \begin{lstlisting}[title={NewOz 2021},language=newoz]
... {
  val x
  val y = 8
  var z = 5
  z = y
}
    \end{lstlisting}
\end{minipage}
\caption{Variables and values syntax comparison}
\label{fig:code-comp-var}
\end{figure}

\subsection{Functions and procedures definition}
In \textit{Oz}, functions are in fact a subclass of procedures.
This allows for a lot of flexibility in the way they are called, since it allows for the ability to store the result in a variable, or to return it directly.
Consequently, it allows for the call to be a statement or an expression, depending on what the situation demands.\newline
However, there is still a syntactic difference between the two : the keywords used to declare them highlights this contrast, and provide an important educational, but also visual, value.
The decision was thus made by M. Mbonyincungu to keep two separate keywords for the two use cases, albeit not the same : "\texttt{fun}" becomes "\texttt{def}" and "\texttt{proc}" becomes "\texttt{defproc}".\newline
More importantly, the way those functions and procedures are called has been completely overhauled, to align with \textit{Scala} and other modern languages.
In particular, functions are now called using their (non-capitalized) name, followed by the comma-separated arguments list enclosed in parentheses.
Similarly, their definition also requires the arguments to be separated by commas.
This feels way better than the old curly braces-based method, on top of improving consistency with the objects syntax (see later).\newline
A comparison of the successive versions of this syntax can be found in Figure~\ref{fig:code-comp-fun}.
\begin{figure}
    \noindent\begin{minipage}{.49\textwidth}
\begin{lstlisting}[title={Oz},language=oz]
local F P in
  proc {P X Y} Y = 2*X end
  fun {F X} 2*X end
  {Browse {F 1}}
end
\end{lstlisting}
    \end{minipage}
    \hfill
    \noindent\begin{minipage}{.49\textwidth}
\begin{lstlisting}[title={Scala/Ozma},language=scala]
... {
  def p(x:Int, y:Int):Unit => {y = 2*x}
  def f(x:Int):Int => {return 2*x}
  println(f(1))
}
\end{lstlisting}
    \end{minipage}
    \begin{center}
    \noindent\begin{minipage}{.89\textwidth}
\begin{lstlisting}[title={NewOz 2020/2021},language=newoz]
... {
  defproc p(x, y) {y = 2*x}
  def f(x) {2*x}
  browse(f(1))
}
\end{lstlisting}
    \end{minipage}
    \end{center}
\caption{Functions and procedures syntax comparison}
\label{fig:code-comp-fun}
\end{figure}

\subsection{Data structures}
The syntactic elements related to data structures like lists, records, and tuples haven't seen many changes compared to \textit{NewOz 2020}.
Lists can now be defined using a square brackets-enclosed, comma-separated list of values, in a desire to align with the way functions and procedures arguments are now defined (see before).\newline

Numeric labels (implicit or not) for records, tuples, and trees are fully supported in \textit{NewOz}, which is something that wasn't possible in \textit{Scala} but existed in \textit{Oz}.
Finally, those labels must begin in \textit{NewOz} with an apostrophe, in order to distinguish them from method calls.\newline
A comparison of the successive versions of this syntax can be found in Figure~\ref{fig:code-comp-data}.
\begin{figure}[t]
    \noindent\begin{minipage}{.49\textwidth}
        \begin{lstlisting}[title={Oz},language=oz]
local L1 L2 L3 R T A in
  L1 = 3|5|1|nil
  L2 = [3 5 1]
  L3 = L1 | L2
  % Numeric labels are implicit
  R = l(a:10 10)
  A = l.1

  T = 1#2#3
end
        \end{lstlisting}
    \end{minipage}
    \hfill
    \noindent\begin{minipage}{.49\textwidth}
        \begin{lstlisting}[title={Scala/Ozma},language=scala]
... {
  val l1 = 3::5::1::Nil
  val l2 = List(3,5,1,Nil)
  val l3 = l1 :: l2
  // Numeric labels are not supported
  case class L(a:Int, b:Int)
  var c = L(10,10)
  val a = c.b
  val t = (1,2,3)
}
        \end{lstlisting}
    \end{minipage}
    \begin{center}
    \noindent\begin{minipage}{.89\textwidth}
        \begin{lstlisting}[title={NewOz 2020/2021},language=newoz]
... {
  val l1 = 3::5::1::nil
  val l2 = [3,5,1]
  val l3 = l1 :: l2
  // Numeric labels are supported
  var c = 'l('a:10, 10)
  val a = c.1
  val t = (1#2#3)
}
        \end{lstlisting}
    \end{minipage}
    \end{center}
\caption{Data structures syntax comparison}
\label{fig:code-comp-data}
\end{figure}

\subsection{Mathematical elements}
All mathematical operations in \textit{NewOz} are directly inherited from \textit{Scala}.
In particular, the keywords "\texttt{orelse}" and "\texttt{andthen}" in \textit{Oz} have been replaced by the more modern "\texttt{||}" and "\texttt{\&\&}", since the confusion with the pipe symbol "\texttt{|}" has been lifted following the syntactic changes made to lists (see above).
Some other weird particularities of \textit{Oz} were corrected, like the comparisons operators "\texttt{=<}" now being replaced by "\texttt{<=}", or the minus prefix-operator finally using the minus "\texttt{-}" sign instead of a tilde "\texttt{\textasciitilde}".\newline
A comparison of the successive versions of this syntax can be found in Figure~\ref{fig:code-comp-math}.
\begin{figure}[t]
    \noindent\begin{minipage}{.49\textwidth}
                 \begin{lstlisting}[title={Oz},language=oz]
local
  A=0.0 B C D Maximum
in
  C = {Sin A}
  D = ~1.2
  B = (D=<C)
  if (({IsFloat D} andthen {IsFloat C}) orelse {Not B}) then
    Maximum = {Max D C}
  end
end
                 \end{lstlisting}
    \end{minipage}
    \hfill
    \noindent\begin{minipage}{.49\textwidth}
                 \begin{lstlisting}[title={Scala/Ozma},language=scala]
... {
  val a=0.0; val b:Any;
  val d:Any; val maximum:Any
  val c = Math.sin(a)
  d = -1.2
  b = (d<=c)
  if (d isInstanceOf Float && c isInstanceOf Float) || !b) {
    maximum = Math.max(d, c)
  }
}
                 \end{lstlisting}
    \end{minipage}
    \noindent\begin{minipage}{.49\textwidth}
     \begin{lstlisting}[title={NewOz 2020},language=newoz]
... { //the library methods used weren't actually available in NewOz 2020
  var a=0.0, b, d, maximum
  val c = sin(a)
  d = ~1.2
  b = (d=<c)
  if ((isFloat(d) && isFloat(c)) || isInt(a)) {
    maximum = max(d, c)
  }
}
     \end{lstlisting}
    \end{minipage}
    \hfill
    \noindent\begin{minipage}{.49\textwidth}
     \begin{lstlisting}[title={NewOz 2021},language=newoz]
... { //the library methods used here are available in NewOz 2021
  var a=0.0, b, d, maximum
  val c = sin(a)
  d = -1.2
  b = (d<=c)
  if ((isFloat(d) && isFloat(c)) || isInt(a)) {
    maximum = max(d, c)
  }
}
     \end{lstlisting}
    \end{minipage}
\caption{Basic operations syntax comparison}
\label{fig:code-comp-math}
\end{figure}

\subsection{Lambdas}
Another element that underwent heavy changes was the way \textit{NewOz 2020} handled lambda functions and procedures.
As M. Mbonyincungu duly notes, lambdas are the same concept as what \textit{Oz} calls anonymous functions and procedures;
but in this case, we feel like the syntax proposed in \textit{NewOz 2020} sacrifices usability, readability, and the respect of \textit{Oz}'s philosophy for the sheer will of bringing the syntax closer to that of \textit{Scala}.
As can be seen in the "Fibonacci" example in appendix~\ref{sec:appendix-examples}, \textit{NewOz 2020}'s notation uses a "\texttt{=>}" like Scala or JavaScript for lambda functions.
Lambda procedures, on the other hand, omit this symbol.
We feel like this is not a very great way to differentiate functions and procedures in this case, because it makes the definition of lambda procedures confusing;
it is our opinion that keeping the keywords "\texttt{fun}" and "\texttt{proc}", or rather their replacement "\texttt{def}" and "\texttt{defproc}", would be preferable.\newline
We also think that this "\textit{arguments} \texttt{=>} \textit{body}" construction, while it fits vey well in \textit{Scala}'s overall syntax, felt a little out-of-place in \textit{NewOz}, giving the feeling that it was a syntactic sugar for something else.
For those reasons, we proposed a solution that was way closer to \textit{Oz}'s original syntax, but that still incorporates the major improvements that the new functions/procedures definition, and the revamped code blocks, represent.\newline
A comparison of the successive versions of this syntax can be found in Figure~\ref{fig:code-comp-lambda}.
\begin{figure}
    \noindent\begin{minipage}{.49\textwidth}
                 \begin{lstlisting}[title={Oz},language=oz]
local Ex1 in
  Ex1 = proc {$ X Y} X*Y end
  {Browse {Ex1 6 4}}
end
                 \end{lstlisting}
    \end{minipage}
    \hfill
    \noindent\begin{minipage}{.49\textwidth}
                 \begin{lstlisting}[title={Scala/Ozma},language=scala]
... {
  val ex2 = (x:Int, y:Int) => x * y
  println(ex2(6, 4))
}
                 \end{lstlisting}
    \end{minipage}
    \noindent\begin{minipage}{.49\textwidth}
                 \begin{lstlisting}[title={NewOz 2020},language=newoz]
... {
  val ex2 = (x, y) => {x * y}
  browse(ex2(6, 4))
}
                 \end{lstlisting}
    \end{minipage}
    \hfill
    \noindent\begin{minipage}{.49\textwidth}
                 \begin{lstlisting}[title={NewOz 2021},language=newoz]
... {
  val ex2 = defproc $ (x, y) {x * y}
  browse(ex2(6, 4))
}
                 \end{lstlisting}
    \end{minipage}
\caption{Lambdas syntax comparison}
\label{fig:code-comp-lambda}
\end{figure}

\subsection{Object-oriented features}
The syntax elements linked to object-oriented programming haven't seen many changes.
The way of accessing class attributes has been adapted to match the changes discussed above regarding mutable variables;
the motivation for this was of course to keep the language consistent, as class attributes are in fact a syntactic sugar for cells.
The keyword "\texttt{super}", used to reference the parent class, can now omit the name of said class : it is now only mandatory to avoid confusion in multi-inheritance cases.
It will be up to the compiler to enforce the presence of this argument when it is necessary.
This improvement was actually discussed by M. Mbonyincungu in his work, but it was abandoned due to the technical limitations of his Parser (see also chapter~\ref{ch:3}).\newline

Similarly, public methods don't need to be written using an \textit{atomLisp} (using the apostrophe "\texttt{'}") anymore;
this was only done due to the fact that the Parser was stateless, and thus couldn't differentiate public methods from attributes in \textit{NewOz 2020}.
Since the new compiler can now leverage a symbol table, this limitation is lifted and more "standard" function names can be used (again, see chapter~\ref{ch:3}).\newline

In \textit{NewOz 2021}, class methods must now use the keyword "\texttt{defproc}". \textit{NewOz 2020} used "\texttt{def}", which is misleading in our opinion since class methods cannot return values in \textit{Oz};
they are procedures, and we feel like this syntactic change is also a better way to explicit this than the "\texttt{meth}" keyword used in \textit{Oz}.\newline

Finally, object application now longer uses the comma "\texttt{,}", symbol, but is now expressed using a dot "\texttt{.}" instead, which feels way more natural to anyone that has used other languages with object-oriented capabilities.\newline
A comparison of the successive versions of this syntax can be found in Figure~\ref{fig:code-comp-obj}.
\begin{figure}
    \noindent\begin{minipage}{.49\textwidth}
\begin{lstlisting}[title={Oz},language=oz]
class Counter
  attr value pm:PrivateMethod
  % The private method is now accessible to children through the attribute
  meth inc(I)
    value := @value + I
  end
  meth PrivateMethod(X)
    {Browse X}
  end
  meth incr
    {self inc(1)}
  end
end
class Child from Counter Other
  meth superCall
    Counter,inc(5)
  end
end
\end{lstlisting}
    \end{minipage}
    \hfill
    \noindent\begin{minipage}{.49\textwidth}
\begin{lstlisting}[title={Scala/Ozma},language=scala]
class Counter {
  var value:Int
  val pm:Unit = privateMethod

  def inc(i) = {
    value = value + i
  }
  private def privateMethod(val x:Int)={
    println(x)
  }
  def incr() = {
    this.inc(1)
  }
}
class Child extends Counter with Other {
  def superCall() = {
    super.inc(5)
  }
}
\end{lstlisting}
    \end{minipage}
    \noindent\begin{minipage}{.49\textwidth}
\begin{lstlisting}[title={NewOz 2020},language=newoz]
class Counter {
  attr value;
  attr pm = PrivateMethod
  def 'inc(i) {
    value := @value + i
  }
  def PrivateMethod(x) {
    browse(x)
  }
  def incr() {
    this.inc(1)
  }
}
class Child extends Counter, Other {

  def superCall() {
    super(Counter).inc(5)
  }
}
\end{lstlisting}
    \end{minipage}
    \hfill
    \noindent\begin{minipage}{.49\textwidth}
\begin{lstlisting}[title={NewOz 2021},language=newoz]
class Counter {
  attr value
  attr pm = PrivateMethod
  def inc(i) {
    value = value + i
  }
  def PrivateMethod(x) {
    browse(x)
  }
  def incr() {
    this.inc(1)
  }
}
class Child extends Counter, Other {
  def superCall() {
    super(Counter).inc(5)
    //In case of single inheritance :
    super.inc(5)
  }
}
\end{lstlisting}
    \end{minipage}
\caption{Classes and objects syntax comparison}
\label{fig:code-comp-obj}
\end{figure}

\subsection{Conditions and pattern-matching}
\textit{NewOz 2021} enforces the presence of a code block in the second part of a match structure (that is, the part after the \texttt{=>} symbol).
This used to be optional in cases where the \textit{consequence} only contained one statement or expression.
However, we felt like this was kind of arbitrary, and we valued the consistency with the conditional structures - in which a proper code block with curly brackets is also mandatory - over this small quality-of-life improvement in switch-case patterns.
We also feel, even though this could be a matter of personal opinions, that a code block makes the code easier to read.\newline

Similarly, the \texttt{catch} clauses make use of pattern-matching on the caught expression.
Their syntax has also been adapted to be consistent with what was discussed above, following \textit{Oz}'s intention of making those two structures as similar as possible.\newline
A comparison of the successive versions of this syntax can be found in Figure~\ref{fig:code-comp-cond}.
\begin{figure}
    \noindent\begin{minipage}{.49\textwidth}
\begin{lstlisting}[title={Oz},language=oz]
local L=[1 3 5] in
  if {Contains L 3} then {Browse 'Has 3'}
  elseif {Contains L 5} then
    {Browse 'Has 5 but no 3'}
  end
  %
  case L
  of 1|L2 then {Browse 'Case 1'}
  [] 3|L2 then {Browse 'Case 2'}
  else {Browse 'Default case'}
  end
end
\end{lstlisting}
    \end{minipage}
    \hfill
    \noindent\begin{minipage}{.49\textwidth}
\begin{lstlisting}[title={Scala/Ozma},language=scala]
val l = List(1, 3, 5)
if (l.contains(3)) {println("Has 3")}
else if (l.contains(5)) {
  println("Has 5 but not 3")
} // This code is not quite equivalent because Scala lists aren't recursive structures :
l(1) match {
  case 1 => println("Case 1")
  case 3 => {println("Case 2")}
  case _ => printlnt("Default case")
}
\end{lstlisting}
    \end{minipage}
    \noindent\begin{minipage}{.49\textwidth}
\begin{lstlisting}[title={NewOz 2020},language=newoz]
var l = [1, 3, 5]
if (contains(l, 3)) {browse("Has 3")}
else if (contains(l, 5)) {
  browse("Has 5")
} //
match l {
  case 1::l2 => browse("Case 1")
  case 3::l2 => {browse("Case 2")}
  else browse("Default case")
}
\end{lstlisting}
    \end{minipage}
    \hfill
    \noindent\begin{minipage}{.49\textwidth}
\begin{lstlisting}[title={NewOz 2021},language=newoz]
var l = [1, 3, 5]
if (contains(l, 3)) {browse("Has 3")}
else if (contains(l, 5)) {
  browse("Has 5")
} //
match l {
  case 1::l2 => {browse("Case 1")}
  case 3::l2 => {browse("Case 2")}
  else {browse("Default case")}
}
\end{lstlisting}
    \end{minipage}
\caption{Conditional structures syntax comparison}
\label{fig:code-comp-cond}
\end{figure}

\subsection{Other structures}
Other syntactic structures have undergone some changes, whose motivation was simply to make them match with what had been modified before.
In particular, the syntax for threads, locks, and the "\texttt{raise}" statement was slightly adapted to incorporate the new scope system using curly braces.\newline
A comparison of the successive versions of this syntax can be found in Figure~\ref{fig:code-comp-other}.
\begin{figure}
    \noindent\begin{minipage}{.49\textwidth}
                 \begin{lstlisting}[title={Oz},language=oz]
local A B E in
  thread {Compute A @B} end
  lock A then C in C=@B B:=C+1 end

  raise E end
end
                 \end{lstlisting}
    \end{minipage}
    \hfill
    \noindent\begin{minipage}{.49\textwidth}
                 \begin{lstlisting}[title={Ozma},language=scala]
... {
  thread { compute(a, b) }
  // Locks do not have nice such a nice syntactic support in Scala
  throw e;
}
                 \end{lstlisting}
    \end{minipage}
    \begin{center}
    \noindent\begin{minipage}{.89\textwidth}
\begin{lstlisting}[title={NewOz 2020/2021},language=newoz]
... {
  thread { compute(a, b) }
  lock (a) {
    var c
    c = b
    b = c+1
  }
  raise {e}
}
\end{lstlisting}
    \end{minipage}
    \end{center}
\caption{Threads, locks and exception raising syntax comparison}
\label{fig:code-comp-other}
\end{figure}

\subsection{Built-in library}
Another important improvement was the import of the complete \textit{Oz} standard library, as described in the official Mozart documentation\footnote{Mozart's online documentation provides an overview of what is called the \textit{Oz Base Environment}, which is an extensive list of functions and procedures directly available to the programmer when writing \textit{Oz} code. This library can be found online at~\cite{mozart2doc}}, into the compiler itself.
The Parser from M. Mbonyincungu only supported a subset of pre-defined functions that were put in manually;
this adaptation will allow a more convenient use of the language by developers, and is a major step towards the goal of reaching functional parity between \textit{NewOz} and \textit{Oz} in the future.

\section{In the end : a self-evaluation}\label{sec:ch2-evaluation}
\textcolor{red}{[TODO] Briefly, what we think of our syntax at this stage}