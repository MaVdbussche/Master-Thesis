%! Author = Martin Vandenbussche

In this chapter, we will describe the general objectives we felt were important to attain with the \textit{NewOz} syntax, as well as the characteristics that we deemed desirable for this syntax to have.
We will then review the important changes that were made  with respects to \textit{NewOz 2020}, and explain the motivation behind said changes.
The goal here is not to repeat what was said before by M. Mbonyincungu in~\cite{jpthesis} : we will instead focus on syntax elements that were either overlooked in that thesis, or that have been significantly modified during this year's work.
Finally, we will conclude the chapter by evaluating whether this new version of \textit{NewOz} fulfills its announced objectives, and outline potential improvements areas that we identified at that stage of the work.\newline

TODO - go once through the whole EBNF to be sure everything is covered !

\section{Our purpose : the big picture}\label{sec:ch2-goal}

\section{In practice : a review of the relevant syntax elements}\label{sec:ch2-review}

\section{In the end : a self-evaluation}\label{sec:ch2-evaluation}