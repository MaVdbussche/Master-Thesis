%! Author = Martin Vandenbussche

- Describe the project, and the motivation behind the new grammar for Oz. (Use Nozc's README's background section as starting point).\newline
- Previous works done in this context\newline
In this chapter, we will provide an overview of what has been achieved in previous works covering Oz's grammar in general, and NewOz in particular
We will describe how those served as a starting point for our reflexions.\newline
\begin{itemize}
    \item NewOz (Jean-Pacifique's thesis)
    \item Ozma
    \item Other previous works on Oz original grammar (used to get the philosophy behind it)
\end{itemize}

\section{NewOz (2020 version)}\label{sec:ch1JPNewOz}
Last year's work of Jean-Pacifique Mbonyincungu used a very systematic approach consisting of reviewing a lot of languages features and syntax elements of Oz.
For each of these, code snippets in both Oz and Scala/Ozma were provided and compared.
The code served as a basis for the discussion that followed, comparing pros and cons of both syntax's, and motivating the final choices being made.
The process was rationalized by using a set of objective factors, allowing to rate each choice on a numeric scale in an attempt to provide the best syntax for each language feature.
\newline
The two main results of this work were the definition of a new syntax, described in~\ref{sec:appendix-a} as an EBNF grammar, as well as the writing of a Parser, which is able to convert any code written in NewOz to the equivalent Oz code.

\subsection{Current state of the NewOz grammar}\label{subsec:ch1CurrentGrammar}
Talk about pros and cons of Jean-Pacifique's NewOz

\subsection{The NewOz Parser}\label{subsec:ch1CurrentParser}
Talk about pros and cons of Jean-Pacifique's Scala Parser

\section{Ozma : a Scala extension}\label{sec:ch1Ozma}
Why this work proved that Oz's philosophy could be applied in other languages and fit nicely in their syntax;
How it laid the foundations of NewOz's Scala-inspired grammar

\section{Other works}\label{sec:ch1OtherWorks}
\begin{itemize}
    \item Kornstaedt 1996
    \item History of the Oz Multiparadigm Language
    \item Concepts, Techniques and Models of Computer Programming (does it fit here ?)
\end{itemize}