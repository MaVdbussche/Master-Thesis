%! Author = Martin Vandenbussche
\begin{lstlisting}[language=newoz,label={lst:compilation-in},title={Input code in \textit{NewOz}}]
declare {
  val a, b=3, c
  def tripleSum(x, y, z) {
    x+y+z
  }
  browse(tripleSum(a,b,c))
}
\end{lstlisting}
\begin{lstlisting}[label={lst:compilation-h},title={Output of the command "\texttt{\$ nozc -h}", showing help regarding the usage of the command}]
(*\texttt{\textcolor{mauve}{nozc 0.0.5-beta}}*)
(c) Martin "Barasingha" Vandenbussche 2021
Run via Picocli 4.6.2-SNAPSHOT
JVM: 16.0.1 (Private Build OpenJDK 64-Bit Server VM 16.0.1+9-Ubuntu-1)
OS: Linux 5.11.0-18-generic amd64
This software is distributed under the BSD license, available at https://github.com/MaVdbussche/nozc/blob/master/LICENSE
(*\texttt{\underline{Usage:}}*)

nozc [(*\texttt{\textcolor{mauve}{-ahpstV}}*)] [(*\texttt{\textcolor{mauve}{--[no-]compile}}*)] [(*\texttt{\textcolor{mauve}{--[no-]keep}}*)] [(*\texttt{\textcolor{mauve}{-d}}*)=<destDirectory>]
     [(*\texttt{\textcolor{mauve}{-o}}*)=<outputFileName>] [(*\texttt{\textcolor{mauve}{-v}}*)=<verbosity>] (*\texttt{\textcolor{mauve}{FILE}}*)...

(*\texttt{\underline{Description:}}*)

Compiles a NewOz program file, or translate it to a valid equivalent Oz program
file.

(*\texttt{\underline{Parameters:}}*)
      (*\texttt{\textcolor{mauve}{FILE}}*)...          The .noz file(s) to compile or translate.

(*\texttt{\underline{Options:}}*)
      (*\texttt{\textcolor{mauve}{--[no-]keep}}*)      Keep the intermediary Oz files in the output folder.
                         True by default
      (*\texttt{\textcolor{mauve}{--[no-]compile}}*)   Continue the program after Oz code generation. The
                         actual Oz code compilation will be executed. True by
                         default
  (*\texttt{\textcolor{mauve}{-t}}*), (*\texttt{\textcolor{mauve}{--tokenize}}*)       Tokenize the NewOz input, print the tokens to STDOUT,
                         and then stop the compilation
  (*\texttt{\textcolor{mauve}{-s}}*), (*\texttt{\textcolor{mauve}{--scan}}*)           Scan/parse the NewOz input, print the AST to STDOUT, and
                         then stop the compilation
  (*\texttt{\textcolor{mauve}{-p}}*), (*\texttt{\textcolor{mauve}{--preAnalyze}}*)     Pre-analyze the NewOz input, print the AST to STDOUT,
                         and then stop the compilation
  (*\texttt{\textcolor{mauve}{-a}}*), (*\texttt{\textcolor{mauve}{--analyze}}*)        Analyze the NewOz input, print the AST to STDOUT, and
                         then stop the compilation
  (*\texttt{\textcolor{mauve}{-v}}*), (*\texttt{\textcolor{mauve}{--verbosity}}*)=<verbosity>
                       The verbosity you want to see in output.
                       Available levels : [OFF, FATAL, ERROR, WARN, INFO,
                         DEBUG, ALL] (default: INFO)
  (*\texttt{\textcolor{mauve}{-o}}*), (*\texttt{\textcolor{mauve}{--out}}*)=<outputFileName>
                       Name of the output file (WITHOUT ANY EXTENSION !). This
                         option will be ignored if you pass more than one input
                         file.
  (*\texttt{\textcolor{mauve}{-d}}*), (*\texttt{\textcolor{mauve}{--directory}}*)=<destDirectory>
                       Output directory for compiled and/or translated files
                         (default: .)
  (*\texttt{\textcolor{mauve}{-h}}*), (*\texttt{\textcolor{mauve}{--help}}*)           Show this help message and exit.
  (*\texttt{\textcolor{mauve}{-V}}*), (*\texttt{\textcolor{mauve}{--version}}*)        Print version information and exit.

(*\texttt{\underline{Exit codes:}}*)
  0   Successful program execution
  1   Exception occurred during program execution
  2   Invalid input (usage)
\end{lstlisting}
\begin{lstlisting}[label={lst:compilation-t},title={Output of the command "\texttt{\$ nozc code.nozc -t}", demonstrating the first part of the compilation (lexer/tokenizer)}]
nozc 0.0.5-beta
(c) Martin "Barasingha" Vandenbussche 2021
Running via Picocli 4.6.2-SNAPSHOT
JVM: 16.0.1 (Private Build OpenJDK 64-Bit Server VM 16.0.1+9-Ubuntu-1)
OS: Linux 5.11.0-18-generic amd64
This software is distributed under the BSD license, available at https://github.com/MaVdbussche/nozc/blob/master/LICENSE
[INFO] : Compiling 1 NewOz file(s) to destination directory "."
[INFO] : Created output file ./code.oz
[INFO] : ==========Scanning started==========
1	 : "declare" = declare
1	 : "{" = {
2	 : "val" = val
2	 : <VARIABLE> = a
2	 : "," = ,
2	 : <VARIABLE> = b
2	 : "=" = =
2	 : <INT> = 3
2	 : "," = ,
2	 : <VARIABLE> = c
3	 : "def" = def
3	 : <VARIABLE> = tripleSum
3	 : "(" = (
3	 : <VARIABLE> = x
3	 : "," = ,
3	 : <VARIABLE> = y
3	 : "," = ,
3	 : <VARIABLE> = z
3	 : ")" = )
3	 : "{" = {
4	 : <VARIABLE> = x
4	 : "+" = +
4	 : <VARIABLE> = y
4	 : "+" = +
4	 : <VARIABLE> = z
5	 : "}" = }
6	 : <VARIABLE> = browse
6	 : "(" = (
6	 : <VARIABLE> = tripleSum
6	 : "(" = (
6	 : <VARIABLE> = a
6	 : "," = ,
6	 : <VARIABLE> = b
6	 : "," = ,
6	 : <VARIABLE> = c
6	 : ")" = )
6	 : ")" = )
7	 : "}" = }
7	 : <EOF> =
[INFO] : ==========Scanning done in 0m:0s:6ms:157µs:827ns==========
\end{lstlisting}
\begin{lstlisting}[label={lst:compilation-s},title={Output of the command "\texttt{\$ nozc code.nozc -s}", demonstrating the second part of the compilation (scanner/syntax analysis)}]
nozc 0.0.5-beta
(c) Martin "Barasingha" Vandenbussche 2021
Running via Picocli 4.6.2-SNAPSHOT
JVM: 16.0.1 (Private Build OpenJDK 64-Bit Server VM 16.0.1+9-Ubuntu-1)
OS: Linux 5.11.0-18-generic amd64
This software is distributed under the BSD license, available at https://github.com/MaVdbussche/nozc/blob/master/LICENSE
[INFO] : Compiling 1 NewOz file(s) to destination directory "."
[INFO] : Created output file ./code.oz
[INFO] : ==========Scanning started==========
[INFO] : ==========Scanning done in 0m:0s:2ms:481µs:399ns==========
[INFO] : ==========Parsing input==========
<InteractiveStatement line="1">
<StatementBlock line="1">
  <ConstantDeclaration line="2" name="a">
    <Value : none>
  </ConstantDeclaration>
  <ConstantDeclaration line="2" name="c">
    <Value : none>
  </ConstantDeclaration>
  <ConstantDeclaration line="2" name="b">
    <Value :>
    <Literal image="3">
  </ConstantDeclaration>
  <FunctionDeclaration line="3" name="tripleSum">
    <Argument>
      <Variable>
        <name:x constant:true>
      </Variable>
    </Argument>
    <Argument>
      <Variable>
        <name:y constant:true>
      </Variable>
    </Argument>
    <Argument>
      <Variable>
        <name:z constant:true>
      </Variable>
    </Argument>
    <ExpressionBlock line="3">
      <BinaryExpression line="4" type="" operator="+">
        <Lhs>
          <BinaryExpression line="4" type="" operator="+">
            <Lhs>
              <Variable>
                <name:x constant:true>
              </Variable>
            </Lhs>
            <Rhs>
              <Variable>
                <name:y constant:true>
              </Variable>
            </Rhs>
          </BinaryExpression>
        </Lhs>
        <Rhs>
          <Variable>
            <name:z constant:true>
          </Variable>
        </Rhs>
      </BinaryExpression>
    </ExpressionBlock>
  </FunctionDeclaration>
  <CallProcedureStatement line="6" name="browse">
    <Arguments>
      <Argument>
        <CallFunctionExpression line="6" name="tripleSum">
          <Arguments>
            <Argument>
              <Variable>
                <name:a constant:true>
              </Variable>
            </Argument>
            <Argument>
              <Variable>
                <name:b constant:true>
              </Variable>
            </Argument>
            <Argument>
              <Variable>
                <name:c constant:true>
              </Variable>
            </Argument>
          </Arguments>
        </CallFunctionExpression>
      </Argument>
    </Arguments>
  </CallProcedureStatement>
</StatementBlock>
</InteractiveStatement>
[INFO] : ==========Parsing done in 0m:0s:25ms:661µs:614ns==========
\end{lstlisting}
\begin{lstlisting}[label={lst:compilation-a},title={Output of the command "\texttt{\$ nozc code.nozc -a}", demonstrating the third part of the compilation (semantic analysis)}]
nozc 0.0.5-beta
(c) Martin "Barasingha" Vandenbussche 2021
Running via Picocli 4.6.2-SNAPSHOT
JVM: 16.0.1 (Private Build OpenJDK 64-Bit Server VM 16.0.1+9-Ubuntu-1)
OS: Linux 5.11.0-18-generic amd64
This software is distributed under the BSD license, available at https://github.com/MaVdbussche/nozc/blob/master/LICENSE
[INFO] : Compiling 1 NewOz file(s) to destination directory "."
[INFO] : Created output file ./code.oz
[INFO] : ==========Scanning started==========
[INFO] : ==========Scanning done in 0m:0s:1ms:627µs:381ns==========
[INFO] : ==========Parsing input==========
[INFO] : ==========Parsing done in 0m:0s:14ms:894µs:931ns==========
[INFO] : ==========Pre-analyzing input==========
[INFO] : ==========Pre-analyzing done in 0m:0s:9ms:49µs:596ns==========
[INFO] : ==========Analyzing input==========
<InteractiveStatement line="1">
<StatementBlock line="1">
  <ConstantDeclaration line="2" name="a">
    <Value : none>
  </ConstantDeclaration>
  <ConstantDeclaration line="2" name="c">
    <Value : none>
  </ConstantDeclaration>
  <ConstantDeclaration line="2" name="b">
    <Value :>
    <Literal image="3">
  </ConstantDeclaration>
  <FunctionDeclaration line="3" name="tripleSum">
    <Argument>
      <Variable>
        <name:x constant:true>
      </Variable>
    </Argument>
    <Argument>
      <Variable>
        <name:y constant:true>
      </Variable>
    </Argument>
    <Argument>
      <Variable>
        <name:z constant:true>
      </Variable>
    </Argument>
    <ExpressionBlock line="3">
      <BinaryExpression line="4" type="Any" operator="+">
        <Lhs>
          <BinaryExpression line="4" type="Any" operator="+">
            <Lhs>
              <Variable>
                <name:x constant:true>
              </Variable>
            </Lhs>
            <Rhs>
              <Variable>
                <name:y constant:true>
              </Variable>
            </Rhs>
          </BinaryExpression>
        </Lhs>
        <Rhs>
          <Variable>
            <name:z constant:true>
          </Variable>
        </Rhs>
      </BinaryExpression>
    </ExpressionBlock>
  </FunctionDeclaration>
  <CallProcedureStatement line="6" name="browse">
    <Arguments>
      <Argument>
        <CallFunctionExpression line="6" name="tripleSum">
          <Arguments>
            <Argument>
              <Variable>
                <name:a constant:true>
              </Variable>
            </Argument>
            <Argument>
              <Variable>
                <name:b constant:true>
              </Variable>
            </Argument>
            <Argument>
              <Variable>
                <name:c constant:true>
              </Variable>
            </Argument>
          </Arguments>
        </CallFunctionExpression>
      </Argument>
    </Arguments>
  </CallProcedureStatement>
</StatementBlock>
</InteractiveStatement>
[INFO] : ==========Analyzing done in 0m:0s:24ms:247µs:102ns==========
\end{lstlisting}
\begin{lstlisting}[label={lst:compilation-a},title={Resulting \textit{Oz} code, demonstrating the successful compilation process}]
declare
A
B=3
C
fun{TripleSum X Y Z}
X + Y + Z
end
in
{Browse {TripleSum A B C} }
\end{lstlisting}
\textit{The formatting of the output \textit{Oz} code, as we mentioned before, should be one of the main attention points in future improvements of \texttt{nozc}.}
