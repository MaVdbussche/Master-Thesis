%! Author = Martin Vandenbussche
\paragraph{Abstract}
\textit{
The \textit{Oz} programming language has proven over the years its value as a learning and research tool for programming paradigms, in universities around the world.
It has had a major influence on the development of more recent languages, and has functionally stood the test of time.
That being said, its syntax lacks the ability the efficiently use some modern programming paradigms.
The first purpose of this work, building upon last year's thesis of Jean-Pacifique Mbonyincungu, is to design a brand new syntax for \textit{Oz}, that will allow the language to tackle new paradigms, while remaining compatible with the existing Mozart system.
We present \texttt{nozc}, the first compiler for this new syntax, built around these requirements.
We then continue by proposing another approach to the future of Mozart, in the form of conservative extensions to existing programming languages.
This would allow to apply \textit{Oz}'s concepts to an existing syntax and its platform.
Finally, we provide some elements of discussion for the future of the \textit{NewOz} project and its quest for the design of a general-purpose, multi-paradigm language.
}
\vfill
\paragraph{Acknowledgments}
\textit{
I would like to thank the readers of this thesis, Nicolas Laurent, Hélène Verhaeghe, and Martin Henz, for dedicating some of their time to this project.\newline
I would also like show my appreciation to the people who contributed to the discussion over on \emph{GitHub}, especially users \texttt{pnrao} and \texttt{BarghestEPL} for their valuable opinions.\newline
Thirdly, I want to thank my family members and friends who helped me proof-reading this document.\newline
Acknowledgement is also due to my employer, \emph{Belfius Banque SA}, for accommodating my modified schedule during the writing of this thesis. I am deeply grateful for their understanding and support.\newline
Finally, I want to express my gratitude to Peter Van Roy for his dedication and implication in all stages of this project. His passion for \emph{Oz} is beautiful to see.
}