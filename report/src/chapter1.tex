%! Author = Martin Vandenbussche

\section{Context of the thesis and the problem to solve}\label{sec:ch1-context}
The \textit{Oz} programming language is a multi-paradigm language developed, along with its official implementation called Mozart, in the 1990s by researchers from DFKI (the German Research Center for Artificial Intelligence), SICS (the Swedish Institute of Computer Science), the University of the Saarland, UCLouvain (the Université Catholique de Louvain), and others.
It is designed for advanced, concurrent, networked, soft real-time, and reactive applications.
\textit{Oz} provides the salient features of object-oriented programming (including state, abstract data types, objects, classes, and inheritance),
functional programming (including compositional syntax, first-class procedures/functions, and lexical scoping), as well as
logic programming and constraint programming (including logic variables, constraints, disjunction constructs, and programmable search mechanisms).
\textit{Oz} allows users to dynamically create any number of sequential threads, which can be described as dataflow-driven, in the sense that a thread executing an operation will suspend until all needed operands have a well-defined value~\cite{mozart2tutorial}.\newline

Over the years, the \textit{Oz} programming language has been used with success in various MOOCs and university courses.
It's multi-paradigm philosophy proved to be an invaluable strength in teaching students the basics of programming paradigms, in a manner that very few other languages could, thanks to its ability to implement such a variety of concepts in a single unified syntax.
However, it has become obvious over time that said syntax also constitutes a drawback.
In particular, \textit{Oz} has not been updated like other languages have, which is hindering its ability to keep a growing and active community of developers around it.\newline

Building upon this observation, it was decided by Professor Peter Van Roy at UCLouvain in 2019 [TO CONFIRM, WHO and WHEN] that a new syntax would be developed for \textit{Oz}, with the utlimate goal of including this syntax in the official release of Mozart 2.
The objective behind what would later be called \textit{NewOz} is ambitious : bringing the syntax of \textit{Oz} to par with modern programming languages, while keeping alive the philosophy that makes its strength : giving access to a plethora of programming paradigms in a single, coherent environment.
This process has started in 2020, with the master thesis of M. Mbonyincungu~\cite{jpthesis}, who created a first design for the \textit{NewOz} syntax, heavily inspired by \textit{Ozma} and \textit{Scala}.\newline

This thesis continues this work by providing three major results :
(a) the definition of a new, more refined version of \textit{NewOz}'s syntax;
(b) the creation of a complete, robust compiler supporting it;
(c) a broad reflection on the syntax design process as a whole, the mistakes that were made during the \textit{NewOz} project so far, and some ideas that might help future contributors to the cause.\newline

In the following sections, we will provide an overview of what our sources of inspiration have been when designing this new syntax, which results previous works have achieved, and we will give an overview of the contributions that this thesis made to the \textit{NewOz} project in general.

Plus large - s'inscrire dans le futur des languages de programmation : goal of what a definitive multiparadigm language should look like.
\section{Inspirations}\label{sec:ch1-inspirations}

DESCRIBE THE TIMELINE OVER THE YEARS + FUTURE

\subsection{Scala}\label{subsec:ch1-scala}
Lazy capabilities\newline
Functional programming\newline
\newline
Lacking syntax for multi-threaded programming : \textit{Ozma} to the rescue

\subsection{Ozma : a Scala extension}\label{subsec:ch1-ozma}
Why this work proved that \textit{Oz}'s philosophy could be applied in other languages and fit nicely in their syntax;
How it laid the foundations of \textit{NewOz}'s \textit{Scala}-inspired grammar

\subsection{NewOz 2020}\label{subsec:ch1-newoz2020}
Last year's work of Jean-Pacifique Mbonyincungu had as main objective to "create, elaborate and motivate a new syntax"~\cite{jpthesis} for \textit{Oz}, by systematically reviewing a subset of the languages features and syntax elements of \textit{Oz}.
For each of these, code snippets in both \textit{Oz} and \textit{Scala}/\textit{Ozma} were provided and compared.
The code served as a basis for the reflection and ensuing discussion, comparing pros and cons of both existing approaches, conceiving a new one when required, and motivating the final choices being made.
The process was rationalized by using a set of objective factors, allowing to rate each choice on a numeric scale in an attempt to provide the best syntax for each language feature.\newline

This thesis has provided two main results :
\begin{itemize}
    \item The definition of a new syntax (which we will refer to as \textit{NewOz} 2020 in this document), as we said before;
    this syntax can be consulted in the appendices of the thesis\footnote{See the appendix C.2 of last year's thesis~\cite{jpthesis}} in the form of an EBNF grammar.
    This result served as the starting point for the syntax designed in this year's work;
    chapter~\ref{ch:2} describes how we covered syntax elements left untouched last year, on top of further refining other aspects.[meh]
    \item The writing of what we will call the "Parser", which is able to convert code written in \textit{NewOz} to the equivalent \textit{Oz} code.
    This Parser was an important step to bring [legitimacy] to the new syntax, as it allows programmers to actually use the syntax in a real-world context;
    however, it lacked some key functionalities present in most compilers, and wasn't very reliable.
    This eventually lead us to the idea that a new technical implementation of a \textit{NewOz} compiler was necessary, as we will explain in chapter~\ref{ch:3}.
\end{itemize}

\subsection{Other works}\label{subsec:ch1-others}
Used to get a sense of the philosophy behind \textit{Oz}
\begin{itemize}
    \item Kornstaedt 1996
    \item History of the \textit{Oz} Multiparadigm Language
    \item Concepts, Techniques and Models of Computer Programming (does it fit here ?)
\end{itemize}

\section{Contributions}\label{sec:ch1-3}

What were the ambitions at the start ?
Rappel : ceci est un chapitre d'intro
\begin{itemize}
    \item Adaptation to JP's syntax
    \item \textit{NewOz} compiler
    \item Community feedback - Describe process of community feedback gathering
    \item Nous sommes une étape dans travail de longue haleine (long-term project/collaboration) - spread over multiple master's projects
    \item Toward ultimate goal of a new, improved, and accepted syntax for the Oz multi-paradigm language (Multiparadigm = now it is accepted that languages must be multiparadigm-Java has lambdas, Scala is functional-objet, Cloud analytics combine functional, concurrent, and database structure)
    \item Etre honnête - c'est compliqué de faire une syntaxe, les limites de temps rencontrées, pourquoi c'est un process multi-year -> ce qu'on a fait pour pallier à ces limitations "physiques"
    \item Final, definitive way of formulating a multiparadigm language.  Oz was a pioneer, followed by Scala, Ozma, etc., but what will multiparadigm languages look like in the future?  In the future when all languages are multiparadigm. We are making steps toward this - take Oz original ideas but with new syntax inspired by existing languages. Lyric goal : here we made one small step in this (Appollo reference)
\end{itemize}

\section{Conclusions on the new syntax}\label{sec:ch1-4}
Rappel : ceci est un chapitre d'intro